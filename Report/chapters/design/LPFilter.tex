\subsection{Design of low-pass filter}\label{sec:designLPFilter}
As seen in \autoref{subsec:downconversion} a filter to remove a unwanted high frequency component from the signal is needed. The filter should have the following characteristics to remove the unwanted frequency components. 
\begin{itemize}
\item Analogue filter since the filter has to filter out frequencies above \SI{1}{\mega\hertz}
\item Low-pass filter 
\item Maximum passband flatness to reduce altering the signal as much as possible.
\item Cut-off frequency at \SI{0.5}{\mega\hertz} (3 dB attenuation point) as described in \autoref{sec:mixercircuitdesign}.
\item An attenuation of 60 dB at \SI{10}{\mega\hertz} since it is desired to filter out any harmonics.
\item Load resistance of \SI{50}{\ohm}
\end{itemize}
With this cut-off characteristic the \SI{333}{\kilo\hertz} information carrying signal, pass through the filter unaltered, while the unwanted  \SI{2}{\giga\hertz} signal is filtered away. A load resistance of \SI{50}{\ohm} is to ensure good signal integrity and power transfer.

A filter with a flat passband is wanted therefore the Butterworth filter type is chosen. To find the necessary order for such a Butterworth filter \autoref{eq:OrderNecessary} is used \citep{AnagogFilters}. 
\begin{equation} \label{eq:OrderNecessary} 
n = \frac{1}{2\cdot \log_{10}\left( \frac{\omega_{s}}{\omega_{p}} \right)} \cdot \log_{10}\left( \frac{10^{ \left( \frac{\alpha_{s}}{10} \right) }-1}{10^{\left( \frac{\alpha_{p}}{10} \right)}-1}  \right) \addunit{1}
\end{equation}
\startexplain
\explain{n is the necessary order}{\si{1}}
\explain{$\omega_{s}$ is the stopband angel}{\si{\radian\per\second}}
\explain{$\omega_{p}$ is passband angel (Cut-off frequency) }{\si{\radian\per\second}}
\explain{$\alpha_{s}$ is the stopband attenuation}{\si{1}}
\explain{$\alpha_{p}$ is the  passband attenuation}{\si{1}}
\stopexplain

The wanted design characteristics are inserted in \autoref{eq:OrderNecessary} and the needed order of the filter is found by rounding up the value. 
\begin{equation} \label{eq:FindOrderNecessary} 
n = \frac{1}{2\cdot \log_{10}\left( \frac{10 \cdot 10^{6} \cdot 2\pi}{0.5 \cdot 10^{6} \cdot 2\pi} \right)} \cdot \log_{10}\left( \frac{10^{ \left( \frac{60}{10} \right) }-1}{10^{\left( \frac{3}{10} \right)}-1}  \right) = 2.307 \to 3
\end{equation}

Due to the high frequency a passive “ladder” filter realisation is chosen. A circuit diagram of the filter is seen on \autoref{fig:LPFilter_realization}. 
\begin{figure}[h]
    \centering
        \includegraphics[width=0.6\textwidth]{figures/circuits/ladder_realization_circuit}
        \caption{Circuit diagram showing a passive third-order “ladder” filter realisation.}
        \label{fig:LPFilter_realization}
\end{figure}

\subsubsection{Finding the transfer function for the circuit}
Looking at \autoref{fig:LPFilter_realization} two \gls{kcl} equations are made, \autoref{eq:LPFilterKCL1} and \autoref{eq:LPFilterKCL2}. 
\begin{equation} \label{eq:LPFilterKCL1} 
\frac{V_{C2} - V_{in}}{sL_{1}} + \frac{V_{C_{2}} - 0}{\frac{1}{sC_{2}}} + \frac{V_{C_{2}} - V_{out}}{sL_{3}} = 0
\end{equation}
\begin{equation} \label{eq:LPFilterKCL2} 
\frac{V_{out} - 0}{R_{L}} + \frac{V_{out} - V_{C_{2}}}{sL_{3}} = 0
\end{equation}

From \autoref{eq:LPFilterKCL1} and \autoref{eq:LPFilterKCL2} a transfer function for the circuit can be deduced, note that $R_{L}$ is normalized to 1.
\begin{equation} \label{eq:LPFilterTfansfer1}  
\frac{V_{out}}{V_{in}} = H(s) = \frac{1}{(L_1 \cdot C_2 \cdot L_3) \cdot s^{3}+(L_1 \cdot C_2) \cdot s^{2}+(L_1+L_3) \cdot s + 1}
\end{equation}

\subsubsection{General third-order butterworth transfer function}
The poles of a butterworth filter can be found with \autoref{eq:LPFilterPoles} \citep{AnagogFilters}. 
\begin{equation} 
p_{k} = e^{j \cdot \left( \frac{2k -1}{2n} \cdot \pi +\frac{\pi}{2} \right) } \label{eq:LPFilterPoles} 
\end{equation}
\startexplain
\explain{$p_{k}$ is a pole}{1}
\explain{$k$ is the pole index $ | k \in \mathbb{N} [ 1 : n]$}{1}
\explain{$n$ is the filter order}{1}
\stopexplain

Since it is a third-order filter, there is three poles, using \autoref{eq:LPFilterPoles} these three poles are calculated. 
%\begin{equation} \label{eq:LPFilterPole1} 
%p_{1} = e^{j \cdot \left( \frac{2 \cdot 1 -1}{2 \cdot 3} \cdot \pi +\frac{\pi}{2} \right) } = e^{j \cdot \frac{2}{3}\pi}
%\end{equation}
%\begin{equation} \label{eq:LPFilterPole2} 
%p_{2} = e^{j \cdot \left( \frac{2 \cdot 2 -1}{2 \cdot 3} \cdot \pi +\frac{\pi}{2} \right) } = -1 
%\end{equation}
%\begin{equation} \label{eq:LPFilterPole3} 
%p_{3} = e^{j \cdot \left( \frac{2 \cdot 3 -1}{2 \cdot 3} \cdot \pi +\frac{\pi}{2} \right) } = e^{j \cdot \frac{4}{3}\pi}
%\end{equation}
\begin{align} \label{eq:LPFilterPolesFound}
 p_{k} &=
  \begin{cases}
   e^{j \cdot \frac{2}{3}\pi}   & \text{for k = 1} \\
   -1							& \text{for k = 2} \\
   e^{j \cdot \frac{4}{3}\pi}   & \text{for k = 3}
  \end{cases}
\end{align}

The transfer function can be made from the three poles found in \autoref{eq:LPFilterPolesFound}.
\begin{subequations}
\begin{align}  
H(s) &= \frac{1}{(s-p_{1})(s-p_{2})(s-p_{3})} \label{eq:TFfromPoles1} \\
 &= \frac{1}{s^{3} + 2s^{2}+2s+1} \label{eq:TFfromPoles5} 
\end{align}
\end{subequations}

\subsubsection{Comparing the two transfer functions to find the normalized values}
Now two different transfer functions for the filter are made. 
By comparing \autoref{eq:LPFilterTfansfer1} and \autoref{eq:TFfromPoles5}, three equations with three unknown variables are made.
\begin{subequations}
\begin{align}
L_1 \cdot C_2 \cdot L_3 &= 1  \label{eq:LPFilter3equations1} \\
L_1 \cdot C_2 &= 2 \label{eq:LPFilter3equations2} \\
L_1 + L_3 &= 2 \label{eq:LPFilter3equations3}
\end{align}
\end{subequations} 

Solving \autoref{eq:LPFilter3equations1}, \autoref{eq:LPFilter3equations2} and \autoref{eq:LPFilter3equations3} for the values of $L_{1}$, $C_{2}$ and $L_{3}$, yields the normalized component values which is seen in \autoref{tab:NormalizedLPFilertValues}. 
\begin{table}[h]
	\centering
	\caption{Normalized component values for a third-order butterworth "ladder" type filter, as the one shown on \autoref{fig:LPFilter_realization}.}\label{tab:NormalizedLPFilertValues}
	\begin{tabular}{l l }
		\textbf{Component}	& \textbf{Value}	\\ \toprule \rowcolor{lightGrey}
		$L_{1n}$	& \SI{1.5}{\henry}		\\ 
		$C_{2n}$	& $\frac{4}{3}$~\si{\farad} 	\\ \rowcolor{lightGrey}
		$L_{3n}$	& \SI{0.5}{\henry} 		\\ 
		$R_{Ln}$	& \SI{1}{\ohm}			\\
	\end{tabular}
\end{table}

\subsubsection{Scaling the normalized values}
The normalized values need to be both frequency and impedance scaled. The frequency scaling factor is given by \autoref{eq:LPFilterFreqScalingFactor} and the impedance scaling factor is given by \autoref{eq:LPFilterImpScalingFactor} \citep{AnagogFilters}. 
\begin{equation} \label{eq:LPFilterFreqScalingFactor}
k_{f} = \frac{\omega_{p}}{\omega_{pn}} = \frac{2\pi \cdot 500 \cdot 10^{3}}{1}= 1 \cdot 10^{6} \cdot \pi
\end{equation}
\begin{equation} \label{eq:LPFilterImpScalingFactor} 
k_{z} = \frac{R_{L}}{R_{n}} = \frac{50}{1} = 50 
\end{equation}

The actual component values for the filter can be found with two scaling equations. To find the inductor values \autoref{eq:LPFilterCApasitorScaling} is used. 
 \begin{equation} \label{eq:LPFilterCApasitorScaling}
L_{is} = \frac{k_{z}}{k_{f}} \cdot L_{in} \addunit{\henry}
\end{equation}
\startexplain
\explain{$i$ is the index number}{\si{1}}
\explain{$L_{is}$ is a scaled inductor}{\si{\henry}}
\explain{$L_{in}$ is a normalized inductor}{\si{\henry}}
\stopexplain
To find the capacitor values \autoref{eq:LPFilter:inductorScaling} is used.
\begin{equation} \label{eq:LPFilter:inductorScaling} 
C_{is} = \frac{1}{k_{f} \cdot k_{z}} \cdot C_{in} \addunit{\farad}
\end{equation}
\startexplain
\explain{$C_{is}$ is a scaled capacitor}{\si{\farad}}
\explain{$C_{in}$ is a normalized capacitor}{\si{\farad}}
\stopexplain

All the needed equations and values have now been found and the actual component values for the filter can be found, by inserting the found normalized component values from \autoref{tab:NormalizedLPFilertValues} together with the scaling factors found in \autoref{eq:LPFilterFreqScalingFactor} and \autoref{eq:LPFilterImpScalingFactor}, into \autoref{eq:LPFilterCApasitorScaling} and \autoref{eq:LPFilter:inductorScaling}. Due to component tolerances the calculated component values, are not easily available for this project, therefore the closest value available is used. 
\begin{subequations}
\begin{align} 
L_{1s} &= \frac{50}{1 \cdot 10^{6} \cdot \pi} \cdot 1.5  = \SI{23.9}{\micro\henry} \xrightarrow{E6} \SI{22}{\micro\henry} \label{eq:LPFilter:inductorScalingDone2} \\
C_{2s} &= \frac{1}{1 \cdot 10^{6} \cdot \pi \cdot 50} \cdot \frac{4}{3} = \SI{8.5}{\nano\farad} \xrightarrow{E6} \SI{8.2}{\nano\farad}  \label{eq:LPFilter:inductorScalingDone1} \\
L_{3s} &= \frac{50}{1 \cdot 10^{6} \cdot \pi} \cdot 0.5  = \SI{8.0}{\micro\henry} \xrightarrow{E6} \SI{8.2}{\micro\henry} \label{eq:LPFilter:inductorScalingDone3} 
\end{align}
\end{subequations} 

\subsubsection{Simulation}
With the circuit designed and all the component values found, the final circuit is simulated in LT-Spice. The result of the simulation is seen on \autoref{fig:LPFilterSImPlot}. 
\begin{figure}[h!]
    \centering
        \includegraphics[width=\textwidth]{figures/design/LPFilter/LPFilterSimPlot}
        \caption{LT-Spice simulation of the circuit on \autoref{fig:LPFilter_realization} with calculated values. The gray boxes represents the limits introduced by the requirements}
        \label{fig:LPFilterSImPlot}
\end{figure}

It is seen from the simulation on \autoref{fig:LPFilterSImPlot} that the passband is flat. It is seen that the attenuation at \SI{5.15}{\mega\hertz} is 60 dB. Furthermore it is seen that the cut-off frequency is located at \SI{543}{\kilo\hertz}. 

The cut-off frequency is of the realised filter is offset by \SI{43}{\kilo\hertz} compared to the design parameter, but the filter has a sharper cut-off curve. It is seen on \autoref{fig:LPFilterSImPlot} that the filter meets the requirements even after a frequency shift. The filter design is accepted. 