\graphicspath{{figures/design/tracking/}}
\chapter{Tracking}\label{ch:design:Tracking}

In \autoref{sec:TrackingTheDrone} it is chosen to use the phase-difference method to track the done. 
The drone transmits a continous-wave signal for the station to receive at \gls{rf}. Two oscillator circuitries are designed for this purpose as seen on \autoref{fig:tracking_overview} where an overview of the transmitter and the receiver is seen. At least three receiver antennas are needed for calculation of the angle in azimuth and elevation planes. It is chosen to frequency downconvert the signal received by the antennas before sampling on a \gls{psoc}. Three mixers therefore needed. Documentation states that the \gls{psoc} can sample at \SI{1}{MSPS} \cite{datasheet:saradc} and the system is therefore designed under this assumption. The chosen tracking has a field of view of \SI{180}{\degree}, and the tracking system doesn't work proper beyond this range. The tracking module do not include a solution for the initial locating if the target is outside of the field of view.

\begin{figure}[h]
	\centering
	\begin{subfigure}[b]{0.4\textwidth}
		\includegraphics[width=\textwidth]{tracking_tx_overview}
		\caption{Overview of the transmitter circuit.}
		\label{fig:tracking_transmitter_overview}
	\end{subfigure}
	~ 
	\begin{subfigure}[b]{0.35\textwidth}
		\includegraphics[width=\textwidth]{tracking_rx_overview}
		\caption{Overview of the receiver circuit.}
		\label{fig:tracking_receiver_overview}
	\end{subfigure}

	\caption{Overview of the tracking system for one angle. The full system include another receiver, mixer and filter. The signals at $S_a$ and $S_b$ is sampled by a \gls{psoc} and used to calculate the phase-differences between the signals}
	\label{fig:tracking_overview}
\end{figure}

%\todo[author=Mads J]{State that we delimit from a field of view larger than 180 degrees. (No locating).}


\section{Choosing a frequency} \label{sec:choosing_a_frequency}
Based on the size of the mechanical platform it is chosen that the receiver antennas should not be placed more the \SI{200}{\milli\meter} apart. The maximum distance between the two antennas should be half the wavelength, as it was shown in \autoref{sec:detSignalDistanceDifference} \citep{TechReport:Amundson2010,TechReport:DirectionFindingPaper}. The wavelength of a signal can be found with \autoref{eq:wavelength}. Combining this equation with the half wavelength distance of the antennas yields \autoref{eq:frequencyforbeacon}. 
\begin{equation} 
\lambda = \frac{c}{f} \addunit{\meter} \label{eq:wavelength}
\end{equation}
\startexplain
\explain{$\lambda$ is the wavelength}{\si{\meter}}
\explain{$c$ is the speed of light}{\si{\meter\per\second}}
\explain{$f$ is the signal frequency}{\si{\hertz}}
\stopexplain

\begin{align} \label{eq:frequencyforbeacon}
f = \frac{c}{2 \cdot x} \addunit{\hertz} 
\end{align}
\startexplain
\explain{$x$ is distance between the antennas}{\si{\meter}}
\stopexplain

Inserting \SI{200}{\milli\meter} as the maximum distance \autoref{eq:frequencyforbeacon} yields:
\begin{equation}
f \geq \frac{3\cdot 10^8~\si{\meter\per\second}}{2 \cdot \SI{200}{\milli\meter}} = \SI{750}{\mega\hertz}
\end{equation}
The Danish regulation regarding frequency bands specifies a frequency range between \SI{863}{\mega\hertz} and \SI{876}{\mega\hertz} with a few gaps in between available for low power radio transmitters to use without explicit permission \citep{web:RegulationFreq}. The largest continuous band of available frequencies in this range is \SI{863}{\mega\hertz} to \SI{868.6}{\mega\hertz}. It is chosen to use this band with a center frequency of \SI{864}{\mega\hertz}.

These frequency bands have limitations on the allowed power output and maximum duty cycle. For \SI{864}{\mega\hertz} the maximum allowed power is \SI{25}{\milli\watt} and the duty cycle limit is 0.1\% \citep{web:RegulationFreq869}. In this project the duty cycle limit is not considered further as the chosen solution is unlikely to work without constant transmission. In a real world product the system would either have to respect the duty cycle limits or work at another frequency with less restrictive limits.

One such frequency is \SI{2.4}{\giga\hertz}. The primary reasons against using this band is noise from other systems as many consumer devices use this frequency. The other reason is high frequency effects of electronics get worse with increasing frequency leading to higher loss from potential mismatch. 

Given the limited high frequency board layout experience of the authors \SI{864}{\mega\hertz} is chosen. With the frequency chosen oscillator circuitry is designed.

\section{Oscillator circuitry design}\label{sec:designOscillator}
The oscillator circuitry is designed around a \gls{vco}. A \gls{vco} is a device used to provide an \gls{ac} signal at a frequency dependent by a input voltage.


For this project the oscillator MAX2622 is chosen \citep{datasheet:MAX2622}. This oscillator has a bandwidth that goes from \SI{855}{\mega\hertz} to \SI{881}{\mega\hertz}, which matches the chosen frequency. Later it become apparent that the oscillator is temperature sensitive. A change in temperature create a frequency drift in the oscillator as shown on \autoref{fig:temperatureSensivityVco} \citep{datasheet:MAX2622}.
\begin{figure}[h]
	\centering
	\includegraphics[width=0.6\linewidth]{figures/design/CurveOfTemperatureForMax2622}
	\caption{The temperature curves of the \gls{vco} MAX2622 \citep{datasheet:MAX2622}. }
	\label{fig:temperatureSensivityVco}
\end{figure} 

If the oscillator is tuned to output a signal of \SI{864}{\mega\hertz} at \SI{25}{\degreeCelsius}, and the temperature of the system drops to \SI{-40}{\degreeCelsius}, a rough estimation of the drift is \SI{22}{\mega\hertz} according to \autoref{fig:temperatureSensivityVco} from the datasheet. 

Frequency drift is undesirable and can be avoided in different ways. A way to avoid frequency drift is to control the tune voltage of the \gls{vco}  with a \gls{pll} circuit. A \gls{pll} works by having a phase detector, which detects the phase different between output signals of the \gls{vco} and a reference signal. This phase difference is used in a feedback loop to adjust the tuning voltage. A \gls{pll} design, is considered too time consuming for this project, and is therefore discarded for a simpler design.

The frequency can also be adjusted by manually adjusting a potentiometer in the \gls{vco} circuit before each flight. When the frequency changes because of the temperature change the two \gls{vco}s could have to be manually aligned. This is jugded to be manageable in the prototype. Adjusting the potentiometer change the tuning voltage. For this project the simple potentiometer approach is chosen based on its simplicity and availability, even though a \gls{pll} would be the ideal solution. 

\subsection{Oscillator circuit design}\label{sec:oscillatorDesign}
Two oscillator circuits are needed: one on the transmitter and one on the receiver. The oscillator's frequencies should be fixed at \SI{864}{\mega\hertz} as chosen in \autoref{sec:choosing_a_frequency}. 

In order to have the oscillator output the wanted frequency a tune voltage is needed. As mentioned in \autoref{sec:designOscillator}, the tune voltage is controlled by a potentiometer. If the potentiometer can change the tune voltage \SI{1}{\volt}, the output frequency can be varied at least \SI{95}{\mega\hertz}.  

It is chosen to tune the voltage between \SI{0.8}{\volt} and \SI{1.8}{\volt} where the frequency \SI{864}{\mega\hertz} is within the bandwidth. 

A \SI{1}{\kilo\ohm} potentiometer and a voltage divider circuit as seen on \autoref{fig:oscillatorCircuitDiagram} is calculated to tune the voltage. 
\begin{subequations}
\begin{align} 
\SI{0.8}{\volt} &= \SI{5}{\volt} \cdot \frac{R_2}{R_1 + R_2} \\
\SI{1.8}{\volt} &= \SI{5}{\volt} \cdot \frac{R_2 + \SI{1000}{\ohm}}{R_1 + R_2 + \SI{1000}{\ohm}}\\
\implies R_1 &= \SI{2688}{\ohm} \xrightarrow{E96} \SI{2670}{\ohm} \\
\implies R_2 &= \SI{512}{\ohm} \xrightarrow{E96} \SI{511}{\ohm}
\end{align} \label{eq:osciristor}
\end{subequations}

\newpage
The datasheet \citep{datasheet:MAX2622} recommends decoupling capacitors $C2$, $C3$ and $C4$ for better high frequency performance and low noise. $C1$ is decoupling on leg 2 carrying the tuning voltage $V_{tune}$. Decoupling on $V_{tune}$ is needed to minimize noise from the power rail. Any noise on $V_{tune}$ results in frequency noise. No matching is done on the output, as the oscillator is internally matched to \SI{50}{\ohm}. The oscillator is designed as shown on \autoref{fig:oscillatorCircuitDiagram}. 
\begin{figure}[h!]
    \centering
        \includegraphics[width=0.8\textwidth]{figures/design/oscillator/oscillatorCircuitDiagram}
        \caption{The oscillator circuit diagram.}
        \label{fig:oscillatorCircuitDiagram}
\end{figure}

Tests show that the oscillator circuit outputs \SI{-8}{\deci\belm}. With the designed oscillator circuit the antennas are now designed. 

\section{Antenna design}\label{sec:antenna_design}
The tracking system incorporates two types of antennas. A transmitting antenna on the drone and three receiving antennas on the stand. These four antennas are designed and constructed in this section. 

The transmitting antenna is statically mounted on the drone and follows the movement of the drone, therefore an omni-directional antenna is preferred.

The three receiving antennas would preferably have a high directionality. Highly directional antennas give a higher gain the more precisely the station is pointed at the drone. As seen in \autoref{sec:detSignalDistanceDifference} high gain and \gls{snr} lead to decreased pointing error. The station's precision would therefore increase when it is needed the most. 


\subsection{Receiving antennas}
Three receiver antennas have to be designed and constructed at \SI{864}{\mega\hertz} with $S{11}$ parameters arbitrarily chosen of at least \SI{-10}{\deci\bel}, where the loss in the reflected power is less than 10\%. The expected maximum gain of the antennas is $\geq \SI{3}{\deci\beli}$. Since the overall design requires a directional antenna, a patch antenna is chosen. Patch antennas have inherently high directionality. The patch antennas can be very rigid, depending on the choice of substrate, and have a well documented design topology \citep{AntennaTheoryBook}. Since all the materials needed for creating a patch antenna are available, this type is chosen. An illustration of a patch antenna shape is seen on \autoref{fig:GeneralPatchAntenna}. 
\begin{figure}[h]
    \centering
        \includegraphics[width=0.5\textwidth]{figures/design/ReceiverAntenna/GeneralPatchAntenna}
        \caption{An illustration of the shape of a patch antenna.}
        \label{fig:GeneralPatchAntenna}
\end{figure}

To calculate the dimensions of the patch antenna a series of equations from \cite{AntennaTheoryBook} is used. The equations is listed below. The equations are used to calculate the dimensions after some initial values are chosen, such as desired input impedance. The dimensions are calculated through MATLAB using the script in \autoref{Code:MatlabReceiverAntenna}.

The width of the patch antenna can be calculated with \autoref{eq:receiverAntenna1}. 
\begin{align} 
W_{patch} = \frac{c}{2 \cdot f} \cdot \sqrt{\frac{2}{\varepsilon_{r} + 1}} \addunit{\si{\meter}} \label{eq:receiverAntenna1} 
\end{align}
\startexplain
\explain{$W_{patch}$ is the width of the antenna patch}{\si{\meter}}
\explain{$c$ is the speed of light in vacuum}{\si{\meter\per\second}}
\explain{$f$ is the frequency}{\si{\hertz}}
\explain{$\varepsilon_{r}$ is relative permittivity of the material}{\si{\farad\per\meter}}
\stopexplain

The length of the antenna patch can be calculated by inserting Equations \ref{eq:receiverAntenna3} and \ref{eq:receiverAntenna4} into \autoref{eq:receiverAntenna2}. 
\begin{subequations}
\begin{align} 
L_{patch} &= \frac{c}{2 \cdot f \cdot \sqrt{\varepsilon_{reff}}} - 2\cdot \Delta L \addunit{\si{\meter}} \label{eq:receiverAntenna2} \\
\varepsilon_{reff} &= \frac{\varepsilon_{r} + 1}{2} + \frac{\varepsilon_{r} - 1}{2}\cdot \frac{1}{\sqrt{1+ \frac{12\cdot h}{W}}} \addunit{\si{\farad\per\meter}} \label{eq:receiverAntenna3} \\
 \Delta L &= 0.412\cdot h \cdot \frac{\left(\varepsilon_{reff} + 0.3\right)\cdot \left(\frac{W}{h} +0.264\right)}{\left(\varepsilon_{reff} - 0.258\right)\cdot\left(\frac{W}{h} +0.8 \right)} \addunit{\si{\meter}} \label{eq:receiverAntenna4} 
\end{align}
\end{subequations} 
\startexplain
\explain{$L_{patch}$ is the length of the antenna patch}{\si{\meter}}
\explain{$\varepsilon_{reff}$ is the effective relative permittivity of the material}{\si{\farad\per\meter}}
\explain{$\Delta L$ is half the difference between the physical and electrical length of the patch}{\si{\meter}}
\explain{$h$ is the thickness of the substrate}{\si{\meter}}
\stopexplain

To find the antenna strip width, a system of two equations is solved. The two equations are gives by Equations \ref{eq:receiverAntenna5} and \ref{eq:receiverAntenna6}.
\begin{subequations}
\begin{align} 
Z_{c} &= \frac{\frac{120\cdot \pi}{\sqrt{\varepsilon_{strip}}}}{\frac{W_{strip}}{h}+1.393+0.667\cdot \ln\left(\dfrac{W_{strip}}{h}+1.444\right)} \addunit{\si{\ohm}} \label{eq:receiverAntenna5} \\
\varepsilon_{strip} &= \dfrac{\varepsilon_{r}+1}{2}+\frac{\varepsilon_{r}-1}{2}\cdot \left(1+12\cdot \frac{h}{W_{strip}}\right)^{-\frac{1}{2}} \addunit{\si{\farad\per\meter}} \label{eq:receiverAntenna6} 
\end{align}
\end{subequations} 
\startexplain
\explain{$Z_{c}$ is the characteristics impedance of the strip}{\si{\ohm}}
\explain{$\varepsilon_{strip}$ is the effective relative permittivity of the strip}{\si{\farad\per\meter}}
\explain{$W_{strip}$ is the width of the antenna strip}{\si{\meter}}
\stopexplain

To calculate the the length of the two feeding slots, Equations \ref{eq:receiverAntenna8} and \ref{eq:receiverAntenna9} are inserted into \autoref{eq:receiverAntenna7}.
\begin{subequations}
\begin{align} 
L_{feed} &= \arccos\left(\frac{\sqrt{Z_{in}\cdot R_{in}}}{Z_{in}}\right)\cdot \frac{L}{\pi} \addunit{\si{\meter}} \label{eq:receiverAntenna7} \\
Z_{in} &= \frac{1}{2\cdot G1} \addunit{\si{\ohm}} \label{eq:receiverAntenna8} \\  
G1 &= W_{strip} \cdot \left(1-\frac{1}{24}\right) \cdot \frac{\frac{2 \cdot \pi \cdot h}{\lambda}}{120\cdot \lambda} \addunit{\si{\siemens}} \label{eq:receiverAntenna9} 
\end{align}
\end{subequations} 
\startexplain
\explain{$L_{feed}$ is the length of the feed sloths}{\si{\meter}}
\explain{$Z_{in}$ is the input impedance of the antenna}{\si{\ohm}}
\explain{$R_{in}$ is the input resistance of the system}{\si{\ohm}}
\explain{$G1$ is the transformed input admittance}{\si{\siemens}}
\explain{$\lambda$ is the wavelength}{\si{\meter}}
\stopexplain

To make the antenna, different combinations of materials are available during design. This project considers two material combinations, FR4 PCB or a PP (Polypropylene) substrate with copper tape. A comparison of the material are given in \autoref{tab:antennaDesign:materialConstants}. 
\begin{table}[h]
\centering
\caption{Materials properties.}\label{tab:antennaDesign:materialConstants}
\begin{tabular}{l l l}
\textbf{Type}	& \textbf{Relative permittivity ($ \varepsilon_r$)} & \textbf{Substrate height ($h$)} 	\\ \toprule \rowcolor{lightGrey}
FR4 PCB& \SI{4.5}{} & \SI{1.5}{\milli\meter} 	\\
PP with copper tape & \SI{2.2}{} & \SI{3}{\milli\meter}  
\end{tabular}
\end{table}

From analysing Equations \ref{eq:receiverAntenna1} and \ref{eq:receiverAntenna2}, it is seen that the use of FR4 PCB results in a physically smaller antenna. Small receiver antennas are preferred since, three antennas have to be mounted on the antenna stand, therefore FR4 PCB is chosen as the substrate for the antenna. Before the antenna is constructed, the size of the ground plane must be determined. To determine the ground plane size, the arrangement of the antennas must be taken into consideration.

\subsubsection{Arrangement of the receiving antennas}
When deciding how to arrange the receiving antennas on the stand, two methods are considered. The formations are illustrated on \autoref{fig:design:antennaMountingArrangement}.
\begin{enumerate}
\item The two bottom antennas are spaced the distance $l$ apart, and the top antenna is placed in the middle, creating a triangular formation. 
\item The two bottom antennas are spaced the distance $l$ apart, and the top antenna is positioned straight above one of the bottom antennas, creating an L formation.
\end{enumerate}

\begin{figure}[h]
\centering
\includegraphics[width=\textwidth]{figures/design/ReceiverAntenna/antennaMountingArrangement}
\caption{Illustration of the two different arranging solutions. The triangular formation is on the left and the L formation is on the right. The distance $l$ is approximately equal to half the wavelength of the tracking signal in free air, $\frac{\lambda_0}{2}$.}\label{fig:design:antennaMountingArrangement}
\end{figure}

The simple approach is using the L formation. This way the system receives two phase differences, one for azimuth and one for elevation which can be used directly. Using the triangular approach, the system also receives two phase differences, one for azimuth and one for elevation. The phase difference for elevation does however contain an azimuth component which has to be removed.

By using the triangular formation, the weight of the antennas is distributed more evenly, but considering the increased complexity introduced by using the triangular formation, the L formation is assessed as being the better choice.

Using the L formation, the distance between the antennas $l$ has to be less than half the wavelength in air $\frac{\lambda_0}{2}$ as stated in \autoref{PhaseDifferenceDetection}. As big a distance between the antennas as possible is wanted, since the error of the \gls{aoa} estimation is inverse proportional to the distance between the antennas, as stated in \autoref{PhaseDifferenceDetection} by \autoref{eq:phase_doa_error}. The distance between the antennas $l$ is therefore calculated by \autoref{eq:design:antennaDistanceL}, but rounded down to avoid phase ambiguity as mentioned in \autoref{PhaseDifferenceDetection}.

\begin{equation}
l = \frac{\lambda_0}{2} = \frac{c}{2f_t}= \frac{\SI{299792458}{\meter\per\second}}{2\cdot \SI{864}{\mega\hertz}} = \SI{173.49}{\milli\meter} \to \SI{172}{\milli\meter}\label{eq:design:antennaDistanceL}
\end{equation}
\startexplain
\explain{$c$ is speed of light in vacuum}{\si{\meter\per\second}}
\explain{$f_t$ is the frequency of the tracking signal}{\si{\hertz}}
\explain{$l$ is the spacing between the antennas}{\si{m}}
\stopexplain

\subsubsection{Construction and test of the receiving antennas}
The antennas should be spaced $l=\SI{172}{\milli\meter}$ apart. The ground plane width of the two bottom antennas is chosen as $W_{GND}=\SI{172}{\milli\meter}$, so the width of ground plane is the same on each side of the antennas. The length is arbitrarily chosen as $L_{GND}=\SI{153}{\milli\meter}$. The length of the feed strip should not matter as long as the intrinsic impedance is $\SI{50}{\ohm}$ and the length is exactly same for all the antennas. A difference of length in the feed strip results in a phase difference. A feed strip length of $L_{feed} =\SI{33.7}{\milli\meter}$ is chosen so the bottom antennas are centered on the ground plane. Lastly a feed slot width of $W_{slot}=\SI{3}{\milli\meter}$ is arbitrarily chosen based on experience. The ground plane sizes of the top antenna is chosen so the antenna spacing is correct, and the feed strip length is the same.

The rest of the antennas' dimensions are calculated by a Matlab script, which is seen in \autoref{Code:MatlabReceiverAntenna}. Now that all the antenna dimensions are known, the antennas are constructed with a \gls{sma} connector attached. A summery of the actual antenna dimensions and the calculated is given in \autoref{tab:antennaDesign:antennaDimensions}. 

\begin{table}[h!]
	\centering
	\caption{Table showing the calculated and resulting antenna dimensions after construction. The given $W_{GND}$ and $L_{GND}$ values are only for the two bottom antennas. }\label{tab:antennaDesign:antennaDimensions}
	\begin{tabular}{l l l l}
\textbf{Name}		& \textbf{Symbol} & \textbf{Desired value} 	& \textbf{Resultant value} 	\\ \toprule \rowcolor{lightGrey}
Patch width			& $W_{patch}$ & \SI{104.619}{\milli\meter}  	& \SI{104.5}{\milli\meter}		\\ 
Patch length		& $L_{patch}$ & \SI{81.633}{\milli\meter} 		& \SI{81.5}{\milli\meter} 		\\ \rowcolor{lightGrey}
Strip width			& $W_{strip}$  & \SI{2.841}{\milli\meter}  		& \SI{3}{\milli\meter} 		\\
Strip length		& $L_{strip}$  & \SI{33.664}{\milli\meter}  		& \SI{33}{\milli\meter} 		\\\rowcolor{lightGrey}
Feed Slot width			& $W_{Feed}$ 	& \SI{3}{\milli\meter}					& \SI{3}{\milli\meter}  		\\
Feed Slot length			& $L_{Feed}$ & \SI{27.173}{\milli\meter} 	& \SI{13}{\milli\meter} 		\\ \rowcolor{lightGrey}
GND plane width		& $W_{GND}$ 	& \SI{172}{\milli\meter} 						& \SI{172}{\milli\meter}  	\\ 
GND plane length	& $L_{GND}$ 	& \SI{153}{\milli\meter}						& \SI{153}{\milli\meter} 
	\end{tabular}
\end{table}

Since the slot length calculations in \autoref{Code:MatlabReceiverAntenna} are merely approximations, an antenna with a shorter slot length is constructed. The slot length is then cut to fit, using a dremmel while measuring the reflections on a network analyser.

On \autoref{fig:design:antennasReceiverMounted} is a picture of the three constructed antennas, attached to the antenna stand in the L formation.
\begin{figure}[h!]
\centering
\includegraphics[width=0.8\textwidth]{figures/design/ReceiverAntenna/MaximesFavoritePicture}
\caption{The three receiving patch antennas attached to the antenna stand.}\label{fig:design:antennasReceiverMounted}
\end{figure}

With the three receiver antennas now designed and constructed, their performance is measured.

\subsubsection{Measurement of the antennas}
The three constructed antennas S{11} parameters are analysed with a network analyser. The result is seen on \autoref{fig:S11Measurement}. 

\begin{figure}[h!]
    \centering
        \includegraphics[width=\textwidth]{figures/design/ReceiverAntenna/S11Mes}
        \caption{Measurement of the power reflected back for the three receiver antennas.}
        \label{fig:S11Measurement}
\end{figure}

The parameter $S{11}$ is the input port voltage reflection coefficient, and it describes the power of the voltage reflected back to the source by the antenna. A small reflection coefficient is desired because the energy reflected back is not radiated in the antenna.  

From \autoref{fig:S11Measurement} it is seen that the three antennas do not have the exact same curvature. This might be due to processing variations and material tolerances. With the chosen frequency at $\SI{864}{\mega\hertz}$ all three antennas have a S11 parameter of less than $\SI{-10}{\deci\bel}$. However by looking closer at the curvatures, it can be seen that a frequency of $\SI{869}{\mega\hertz}$, is the frequency where the sum of input port voltage reflection coefficients are lowest. At this point antenna 2 has its lowest possible reflection coefficient. Antenna 1 and 3 have an acceptable reflection coefficient, where the worst S11 parameter is $\SI{-22.18}{\deci\bel}$ at $\SI{869}{\mega\hertz}$. 

In order to have the best possible S11 parameters it is chosen to change the desired operation frequency from \SI{864}{\mega\hertz} to \SI{869}{\mega\hertz} to achieve a better reflection coefficient. The frequency \SI{869}{\mega\hertz} is in the allowed band \citep{web:RegulationFreq} and has the same limits on power as found in \autoref{sec:choosing_a_frequency}.  

A test of antenna 2 was conducted to measure realized gain pattern of the patch antennas. Only one antenna is measured, since it is assessed that the antennas have approximately the same propagation patterns. It is tested in anechoic chamber with help of Kristian Bank, engineering assistant from Aalborg University. The results of the test are shown on \autoref{fig:design:antennaReveicersTestPropagation}.

\begin{figure}[h!]
    \centering
    \begin{subfigure}[b]{0.45\textwidth}
        \includegraphics[width=\textwidth]{figures/design/ReceiverAntenna/antennaTransmission2D}
    \end{subfigure}
    ~ 
    \begin{subfigure}[b]{0.45\textwidth}
       \includegraphics[width=\textwidth]{figures/design/ReceiverAntenna/antennaTransmission3D}
    \end{subfigure}
    \caption{Plot of the data, from a measurement of the realized gain of antenna 2, in respect to the spherical coordinates, $\phi$ and $\theta$. The test was done with a \SI{870}{\mega\hertz} signal.}\label{fig:design:antennaReveicersTestPropagation}
\end{figure}

It is seen from \autoref{fig:design:antennaReveicersTestPropagation} that the maximum realized gain is $G_{max} = \SI{0.03}{\deci\beli}$, which is lower than the expected \SI{3}{\deci\beli}. Realized gain is a measure of gain with loss due to reflections included. The overall efficiency of the antenna is measured as 22\%. 
%
%
%Mismatch between the measurement equipment and the antennas have influence on low gain. To get higher gain and efficiency, a match is preferred but not done in this project. 
%
The antennas have a high directionality and approximately \SI{-12}{\deci\beli} realized gain for signals behind the antenna, which correlates with the expected gain pattern. Seeing as the antennas have desired gain pattern, although with low realized gain, it is assessed that the antennas is sufficient for the purpose of the project.



\subsection{Transmitting antenna}
To transmit the tracking signal a single antenna is used. The drone is not always faced towards the antenna and therefore an isotropic antenna is preferred but however that is not possible. An omnidirectional dipole is chosen. 

In this project it is delimited from designing a transmitting antenna and is decided to acquire an already constructed antenna. One antenna with the wanted characteristics is available, W1063 \citep{datasheet:isotropicDipole}. Its characteristics are as listed in \autoref{tab:TransmittingAntennaCharacteristics}.
\begin{table}[h!]
	\centering
	\caption{Electrical specifications of the transmitting antenna.}\label{tab:TransmittingAntennaCharacteristics}
	\begin{tabular}{ll}
		\textbf{Property} & \textbf{Value} \\ \toprule \rowcolor{lightGrey}
		Name 		& W1063 \\
		Weight	& \SI{25.6}{\gram} \\ \rowcolor{lightGrey}
		Frequency range & 868 – 928 \si{\mega\hertz} \\ 
		Gain & \SI{3}{\deci\beli}\\ \rowcolor{lightGrey}
		\gls{swr} & $\leq$ 2 \\ 
	\end{tabular}
	\caption*{\citep{datasheet:isotropicDipole}}
\end{table}

The antenna is tested on a network analyser and a S11 of \SI{-9.5}{\deci\bel} and a \gls{swr} of \SI{2}{}. To conclude the antenna design, a test of transmission between the antennas is conducted.

\subsection{Test of transmission between the antennas}\label{sec:AntennaTest}
Transmission between the transmitting antenna and a receiving antenna is tested in \autoref{appendix:antennaTransmission}. During testing it is observed that the received power follows Friis equation with slight deviations. The test setup is however not free of reflecting surfaces which might greatly affect the results. The receiving antennas have gain patterns as illustrated in \autoref{fig:design:antennaReveicersTestPropagation}, with a maximum realized gain of \SI{0.03}{\deci\beli}. The omni directional transmitter has a maximum gain of \SI{3}{\deci\beli} and a \gls{swr} of less than 2 \citep{datasheet:isotropicDipole}. The maximum power lost to mismatch is \SI{0,51}{\decibel} giving a realised gain of \SI{2,49}{\deci\beli} in the worst conditions. 

The properties of the patch antennas are deemed sufficient for further use.

\newpage
\section{Power transmission} \label{sec:transmitter_power}
To locate the drone it needs to be able to transmit the beacon signal. The signal frequency has been chosen to \SI{869}{\mega\hertz}, the \gls{vco} MAX2622 has been chosen and receiver antennas are patch antennas with 22\% efficiency and \SI{0.03}{\deci\beli} gain measured and the transmitter antenna with \SI{2.49}{\deci\beli} gain.

In \autoref{PhaseDifferenceDetection} \autoref{eq:phase_reqsnr} determined a required \gls{snr} to be $> \SI{46.7}{\decibel}$. The transmitted power that are required to achieve that \gls{snr}, can be calculated using Friis equation and an estimation for the power of the noise. In this project it is delimited from taking noise factor in components used into account. 

\subsection{Power of received noise}
The noise can be estimated by a simple approximation that states, for every \si{\hertz} of bandwidth the power of the noise is \SI{-174}{\deci\belm} \citep{web:NoiseWave}.
The 1 \gls{msps} sampling rate of the \gls{psoc} can't handle the \SI{869}{\mega\hertz} signal, so a low-pass filter is designed to have a cutoff frequency of \SI{500}{\kilo\hertz}. Therefore the bandwidth is \SI{1}{\mega\hertz} since any noise \SI{500}{\kilo\hertz} below the transmitting frequency and \SI{500}{\kilo\hertz} above the transmitting frequency is included in the signal.
The power of the noise can be approximated with \autoref{eq:PhaseDettection:NoisApprox}. 
\begin{subequations}
\begin{align} 
P_{n} &= \SI{-174}{\deci\belm\per\hertz} + 10 \cdot \log_{10}(B) \addunit{\deci\belm} \label{eq:PhaseDettection:NoisApprox}\\
			 &= \SI{-174}{\deci\belm\per\hertz} + 10 \cdot \log_{10}(\SI{1}{\mega\hertz}) = \SI{-114}{\deci\belm} \label{eq:PhaseDettection:NoisApproxFoundValue}
\end{align}
\end{subequations}
\startexplain
\explain{$P_{noise}$ is the power of the noise}{\si{\deci\belm}}
\explain{$B$ is the Bandwidth}{\si{\hertz}}
\stopexplain

\subsection{Power of received signal}
With the power of the noise determined to be \SI{-114}{\deci\belm}, the needed power of the signal can also be determined if \gls{snr} $>\SI{46.7}{\deci\bel}$. This calculation is in \autoref{eq:PhaseDettection:NeededPoserOfSignalFoundValue}. 
	

\begin{equation}
P_{s} \geq \SI{-114}{\deci\belm} + \SI{46.7}{\deci\bel} = \SI{-67.3}{\deci\belm} \label{eq:PhaseDettection:NeededPoserOfSignalFoundValue} 
\end{equation}
\startexplain
\explain{$P_{s}$ is the power of the signal}{\si{\deci\belm}}
\stopexplain

This is the very minimum required signal power, and does not account for the noise in amplifiers and other components.

\subsection{Transmission power}\label{sec:transmissionPower}
From \autoref{eq:PhaseDettection:NeededPoserOfSignalFoundValue} it is seen that a received signal strength of minimum \SI{-67.3}{\deci\belm} is needed to have an acceptable \gls{snr}. A simplified version of Friis transmission equation is used to approximate the needed transmission power in order to receive \SI{-67.3}{\deci\belm}. 
\begin{subequations}
\begin{align} 
P_{r} &= G_tG_r \cdot \left(\frac{\lambda}{4\pi\cdot d}\right)^{2} \cdot P_{t}\label{eq:PhaseDettection:FrissIsAcoolGuy1} \addunit{\watt}\\
P_{t} &= P_{r} \cdot \left(G_tG_r\right)^{-1} \cdot \left(\frac{\lambda}{4\pi\cdot d}\right)^{-2} \label{eq:PhaseDettection:FrissIsAcoolGuy2} \addunit{\watt}
\end{align}
\end{subequations}
\startexplain
\explain{$P_{t}$ is the power of the transmitted signal}{\si{\watt}}
\explain{$P_{r}$ is the power of the received signal}{\si{\watt}}
\explain{$G_t$ is the gain of the transmitter antenna}{\noSIunit}
\explain{$G_r$ is the gain of the receiver antenna}{\noSIunit}
\explain{$\lambda$ is the wavelength of the transmitted signal in the medium, in this case air}{\si{\meter}}
\explain{$d$ is the distance between the receiver and the transmitter}{\si{\meter}}
\stopexplain

Using \si{\decibel} for gain and \si{\deci\belm} as a unit for power \autoref{eq:PhaseDettection:FrissIsAcoolGuy2} becomes \autoref{eq:FriisdB}.
\begin{equation} \label{eq:FriisdB}
P_t = P_r - G_t - G_r - 20\log_{10}\left(\frac{\lambda}{4\pi d}\right) \addunit{\watt}
\end{equation}

The distance is set to be the maximum required \SI{120}{\meter} and a frequency of \SI{869}{\mega\hertz}. 
The realized gain of the receiver antenna is taken to be the maximum gain \SI{0}{\decibel}, as this is when the receiver antennas are directly pointed at the transmitter and it is in this case when the precision is needed most. The transmitting antenna has a gain of \SI{2.49}{\deci\beli} as stated in \autoref{sec:AntennaTest}. \autoref{eq:FriisdB} then gives \autoref{eq:PhaseDettection:NeededTransmisionPower}.
\begin{align} 
P_t \geq \SI{-67.3}{\deci\belm} - \SI{2.49}{\decibel} - \SI{0}{\decibel} - 20\log_{10}\left(\frac{\frac{c}{\SI{869}{\mega\hertz}}}{4\pi \SI{120}{\meter}}\right) = \SI{3.02}{\deci\belm} \label{eq:PhaseDettection:NeededTransmisionPower}
\end{align}

It is found that in order to have an \gls{snr} of \SI{46.7}{\deci\bel} when the antennas are pointed directly at each other, the signal should be transmitted with a power greater than \SI{3.02}{\deci\belm}. 

As mentioned in \autoref{sec:oscillatorDesign} the oscillator circuit outputs \SI{-8}{\deci\belm}. This means that the power has to be amplified so that \autoref{design:eq:transmitterAmplifierRequirement} is fulfilled.
\begin{equation}
- \SI{8}{\deci\belm} + A_{dB} \geq \SI{3.02}{\deci\belm}\label{design:eq:transmitterAmplifierRequirement}
\end{equation} 

It is therefore concluded that the oscillator should be applied an amplification of $A_{dB} \geq \SI{11.02}{\deci\bel}$ to achieve an acceptable \gls{snr}.

\subsection{Amplifier}
An amplifier is designed the oscillator output to achieve an acceptable \gls{snr} and to transmit higher power from the transmitter antenna to the receiver antennas. A circuit with a MAR3 \citep{datasheet:MAR3} and a MSA-1105 \citep{datasheet:MSA1105} is designed, based on the fact that they at frequencies up to \SI{1}{\giga\hertz}, the sum of the amplification is \SI{24.5}{\decibel} and availability during design. 

The transmitter circuit consists of the oscillator, the amplifier and the transmitter antenna attached to a drone with a battery to supply the power. A \SI{9}{\volt} battery is chosen as the power supply. The datasheet has a recommended application circuit and a table with resistor values for optimum bias that are used \citep{datasheet:MAR3}. A diagram of the circuit can be seen on \autoref{fig:TransAmp}. 
\begin{figure}[h]
\centering
\includegraphics[width=0.6\textwidth]{figures/design/TransmitterAntenna/Amplifier}
\caption{A circuit diagram of the amplifier. C1, C2 and C3 block DC. L1 and L2 block AC for the supply voltage.}\label{fig:TransAmp}
\end{figure}

A test of the amplifier in \autoref{appendix:TransmitterInt} shows the amplifier is amplifying \SI{16,18}{\deci\belm} and meets the required \SI{11.02}{\deci\belm}. The loss of the amplification could be due to one of the amplifier being biased incorrectly and reflection within the rest of the circuit. Since the gain is higher then the required the amplification circuit is used. The power received at the receiver is \SI{-62,14}{\deci\belm}.
%
\section{Frequency mixer} \label{sec:tracking_mixer}

A mixer is an electronic circuit that performs a multiplication of two input signals. The symbol of a mixer is seen on \autoref{fig:mixer_symbol}. Typically one of the two inputs is a \glsentrylong{lo} (\glsentryshort{lo}) which oscillates at some fixed predefined frequency.
\begin{figure} [h!]
\centering
\includegraphics[width=0.3\linewidth]{mixer_component}
\caption{Symbol of frequency mixer.}
\label{fig:mixer_symbol}
\end{figure}

Mixers are useful for modulating or demodulating signals. When the mixer's input is a cosine at frequency $f_{RF}$, and the \gls{lo} is a cosine at frequency $f_{LO}$, the output signal contain components of both signals as seen in \autoref{eq:basic_mixer}.
\begin{subequations}\label{eq:basic_mixer}
\begin{align} 
S_o &= \cos \left( {2\pi f_{RF} \cdot t} \right) \cdot \cos \left( {2\pi f_{LO} \cdot t} \right) \addunit{V}  \\
&= \frac{1}{2} \left[ \cos \left( {2\pi (f_{RF} + f_{LO}) t} \right) + \cos \left( {2\pi (f_{RF} - f_{LO}) t} \right) \right] \addunit{V}\label{eq:basic_mixer2}
\end{align}
\end{subequations}
\startexplain
\explain{$S_o$ is the resulting signal}{\si{\volt}}
\explain{$f_{RF}$ is the radio frequency}{\si{\hertz}}
\explain{$f_{LO}$ is the local oscillator frequency}{\si{\hertz}}
\explain{$t$ is the time}{\si{\second}}
\stopexplain


If the frequencies $f_{RF}$ and $f_{LO}$ are equal the result is a cosine of twice the frequency and a DC level. 
\begin{equation}
S_o =\frac{1}{2} \left[ \cos \left( {2\pi (2 \cdot f_{RF}) t} \right) + 1 \right] \addunit{V} \label{eq:basic_mixerDC}
\end{equation}

This property can be used to measure the phase difference between two signals. If one of the signals from \autoref{eq:basic_mixer} contains a phase shift compared to the other, the DC level is directly affected. \autoref{eq:mixer_phase} is an example of this.
\begin{subequations} \label{eq:mixer_phase}
\begin{align} 
S_o &= \cos \left( {2\pi f_{RF} \cdot t + \varphi } \right) \cdot \cos \left( {2\pi f_{RF} \cdot t} \right) \addunit{\volt}  \\
&= \frac{1}{2}\left[ {\cos \left( {2\pi (2 \cdot f_{RF}) t + \varphi } \right) + \cos \left( \varphi  \right)} \right] \addunit{\volt}
\end{align}
\end{subequations} 

The DC level is represented by the constant $\cos \left( \varphi  \right)$. Additionally the high frequency part has the same phase. The output signal is low-pass filtered to remove the high frequency part leaving:
\begin{equation} \label{eq:mixerPhaseDiff1}
	S_{o-DC} = \frac{1}{2}  \cos \left( \varphi  \right) \addunit{\volt}
\end{equation}

The phase difference $\varphi$ can then be extracted as shown in \autoref{eq:mixerPhaseDiff2}.
\begin{equation}
	\varphi = \arccos(2 \cdot S_{o-DC}) \addunit{\radian} \label{eq:mixerPhaseDiff2}
\end{equation}

The theory described above applies for ideal mixers. Mixers are nonlinear devices created from nonlinear devices and therefore carry imperfections. One of the concerns is harmonics. The output signal of a real world mixer contain parts of the original signals as well as harmonics of the inputs. To filter off harmonics a low-pass filter is used.

Real world devices are additionally specified within certain limits. The main limits for the use case of this project are input and output power levels, impedances and frequency limits.

\subsection{Mixer circuit design} \label{sec:mixercircuitdesign}
The system overview on \autoref{fig:tracking_overview} shows the placement of the mixers in the full system. The main requirement for the chosen mixer device is the lower frequency limit. The wanted input frequency is \SI{869}{\mega\hertz} and the wanted output frequency is ideally \SI{0}{\hertz}. The available mixers that can receive above \SI{800}{\mega\hertz} and output DC is limited to more expensive solutions than what was available during the development of this project and a compromise is made. The requirement for the frequency of the output is raised to \SI{500}{\kilo\hertz}, which is the highest frequency signal that the \gls{adc} can sample and still be within the Nyquist sampling theorem. The \gls{adc} is described further in \autoref{sec:adc}.

Secondary requirements to the mixer include input and output power, the mixer gain and the input and output impedance. 
The first concern is the power levels. The oscillator has a specified output power of \SI{-3}{\deci\belm} \citep{datasheet:MAX2622} but the designed oscillator outputs \SI{-8}{\deci\belm}. If the mixers are driven from different oscillators any phase difference between the two oscillator signals directly affect the signal. All the mixers have to run off of the same oscillator and the power is split. To reduce the noise on the \gls{lo} signal the amplification should be reduced as much as possible. These two concerns create a requirement for the mixers' input power level to be within \SI{10}{\decibel} of the output level of the oscillator.

Thirdly, as calculated in \autoref{sec:transmitter_power} the incoming power to the mixers is very low. To minimize the amount of extra components required in the signal path, as much gain as possible is wanted in the mixer.

And last, the inputs and outputs of the mixer should be matchable to \SI{50}{\ohm} to ensure good signal integrity and power transfer.

Based on the requirements, the Linear Technology LT5560 low power active mixer is chosen. The LT5560 is a high-performance broad-band mixer useful for both up- and down-conversion \citep{datasheet:LT5560}. The mixer fulfills the requirement for a low lower \gls{if} limit with the device being specified down to \SI{10}{\kilo\hertz}. The \gls{lo} input power is typically in the range \SI{-6}{\deci\belm} to \SI{1}{\deci\belm} which requires no amplification in the case of one mixer. The mixer has a conversion gain of around \SI{2}{\decibel} at \SI{900}{\mega\hertz} and the gain is quite linear in a range of about \SI{150}{\mega\hertz} around this point. The mixer's datasheet \citep{datasheet:LT5560} has application notes concerning impedance matching, enabling easy adaptation for a \SI{50}{\ohm} matched application at \SI{869}{\mega\hertz}. 

\subsection{Downconversion} \label{subsec:downconversion}
Because of the lower specified frequency limit of \SI{10}{\kilo\hertz} a frequency component on the signal output of the mixer is required. The frequency of the signal after the mixing is known as an \glsentrylong{if} (\glsentryshort{if}). This signal is later mixed again, as mentioned in \autoref{sec:calc_of_phase}.

As specified in \autoref{sec:adc} the \gls{adc} can sample \SI{1}{} \gls{msps}. To get a good number of samples of every wave of the output, the output frequency $\Delta f$ is chosen as $\frac{1}{3}$ of the sampling frequency $f_s$.
\begin{equation}
	\Delta f = \frac{f_s}{3} = \SI{333}{\kilo\hertz}
\end{equation}
The output is shifted to a frequency of $\Delta f$ by making the local oscillator frequency $f_{RF} + \Delta f$. The output becomes a signal with two frequency components as seen in \autoref{eq:mixer_down_convert}. The RF signal contains the phase component in the new lower frequency.
\begin{subequations}
\begin{align}
S_o &= \cos \left( {2\pi f_{RF} \cdot t + \varphi } \right)\cos \left( {2\pi \left( {f_{RF} + \Delta f} \right)t} \right) \addunit{\volt}  \label{eq:mixer_down_convert}  \\ 
 &= \frac{1}{2}\left[{\cos \left( {2\pi \left( {2f_{RF} + \Delta f} \right)t + \varphi } \right) + \cos \left( { 2\pi \Delta f \cdot t - \varphi } \right)} \right] \addunit{\volt} \label{eq:mixer_down_convert2} 
\end{align}
\end{subequations}
As in \autoref{eq:mixer_phase} the signal contains a high frequency components, low frequency components and harmonics, all of which contain the phase information. The high frequency part and harmonics are removed with a low-pass filter leaving only the low frequency \gls{ac} signal, as seen in \autoref{eq:mixer_down_convert_filtered}.
\begin{equation}
S_{o-filtered} = \frac{1}{2}  \cos(2\pi \Delta f \cdot t - \varphi) \addunit{\volt} \label{eq:mixer_down_convert_filtered} 
\end{equation}

This down-conversion is applied to two incoming signals with differing phase contributions. By applying the same \gls{lo} in mixing both signals, any phase difference compared to the oscillator is cancelled out.
The low frequency signals with phase components are later sampled by an \gls{adc} and the phase information in each signal is used. This is covered in detail in \autoref{sec:calc_of_phase}.

\subsection{Impedance matching}
Based on the RF chosen in \autoref{sec:choosing_a_frequency}, the \gls{if} and the characteristics of the mixer given by its datasheet \cite{datasheet:LT5560} the circuitry for impedance matching is constructed. There are three parts to the matching-design: the RF-input, the LO-input and the IF-output. All input and outputs are wanted as single-ended, but all inputs and outputs of the LT5560 are balanced so this is considered as well.

The LO-input is driven single ended, based on the typical application diagram \citep{datasheet:LT5560}. It is chosen to use the typical application and test the real world performance. A matching of the LO-input is seen on \autoref{fig:match_smith}. The circle illustrates an SWR of 2, and everything within is better. The matching is deemed sufficient.

\begin{figure} [h]
\centering
\includegraphics[width=1\textwidth]{mixer_lo_matching_smith_cutout}
\caption{A smith chart cutoff of the LO-input matching. The circle indicate \gls{swr}$<$2, where the reflection in less. The dot indicates the impedance before matching and the cross is the impedance after matching.}
\label{fig:match_smith}
\end{figure}

As three mixers are driven from the same oscillator a power splitter should ideally be used. To simply the design, the return loss of splitting the \SI{50}{\ohm} matched output into the three mixers is accepted and not considered further.

The RF-input matching is a lumped element matching based on the section of the datasheet describing the technique \citep{datasheet:LT5560}. \autoref{fig:lumped_element_matching} shows a balun (Balanced to Unbalanced) connection used in lumped element matching. $R_A$ represents the impedance to match to; in this case \SI{50}{\ohm}. $R_B$ is the differential resistance to match, which in the case of the RF-input is specified as \SI{28,8}{\ohm} at both \SI{760}{\mega\hertz} and \SI{900}{\mega\hertz}. The resistance is assumed linear in the region and the calculation is made for \SI{869}{\mega\hertz}.

\begin{figure} [h]
	\centering
	\includegraphics[width=0.4\textwidth]{lumped_element_matching}
	\caption{Circuit realization for doing lumped element matching \citep{datasheet:LT5560}.}
	\label{fig:lumped_element_matching}
\end{figure}

The calculation of the sizes of the lumped capacitors and inductors are given as
\begin{equation}
	L_0 = \frac{\sqrt{R_A \cdot R_B}}{2\pi f_{RF}} \addunit{\henry}
\end{equation}
and
\begin{equation}
	C_0 = \frac{1}{\sqrt{R_A \cdot R_B} \cdot 2\pi f_{RF}} \addunit{\farad}
\end{equation}
The values for \SI{869}{\mega\hertz} are calculated. Firstly the inductors.
\begin{equation}
	L_0 = \frac{\sqrt{\SI{50}{\ohm} \cdot \SI{28.8}{\ohm}}}{2\pi \cdot \SI{864}{\mega\hertz}} \approxeq \SI{6,95}{\nano\henry} \xrightarrow{E6} \SI{6,8}{\nano\henry}
\end{equation}
Then the capacitors.
\begin{equation}
C_0 = \frac{1}{\sqrt{\SI{50}{\ohm} \cdot \SI{28.8}{\ohm}} \cdot 2\pi \cdot \SI{869}{\mega\hertz}} \approxeq \SI{4.88}{\pico\farad} \xrightarrow{E6} \SI{4,7}{\pico\farad}
\end{equation}


\subsubsection*{Output matching}
When matching the output of the mixer there are two main considerations: matching the output impedance of the mixer and amplifying the signal further.

\begin{figure} [h]
\centering
\includegraphics[width=1\linewidth]{mixer_output}
\caption{Cutout of the mixer schematic with output matching of the LT5560 based on the datasheet \cite{datasheet:LT5560} and additional amplification.}
\label{fig:mixer_output}
\end{figure}

The impedance matching is based on Figure 21 of the datasheet \citep{datasheet:LT5560}. This figure gives an example of how to use an operation amplifier in a differential amplifier setup to make the balanced to unbalanced conversion. The op amp additionally biases the signal level to \SI{2.5}{\volt}, which is half of the power supply voltage. Both output pins need to biased to the supply voltage. R301 and R302 on \autoref{fig:mixer_output} provide this bias and C308 and C309 AC-couple the signal. With the matching established, the signal level is adjusted.

The signal power level received by the mixers is estimated to be at least \SI{-62,14}{\deci\belm} in \autoref{sec:transmitter_power}. The LT5560 has \SI{2}{\decibel} - \SI{3}{\decibel} gain giving an estimated output of at least \SI{-60,14}{\deci\belm}. To get minimal quantization error when measuring, the amplitude of the signal should be as high as possible within the measuring limits of the \gls{adc}. Calculations on the quantization error are expanded in \autoref{subsec:quantization_error}. The signal swing of the output with a power of \SI{-60,14}{\deci\belm} is 
\begin{equation}
	\sqrt{\SI{968}{\pico\watt} \cdot 50~\si{ohm}} \approxeq \SI{0,227}{\milli\volt}_{rms} = \SI{0,321}{\milli\volt}_{amplitude}
\end{equation}

This is a very low voltage swing compared to the \SI{2}{\volt} range of the \gls{adc} and amplification is wanted to get better resolution. 
The requirements for the design are based on the \gls{adc} specifications. The range of the \gls{adc} on the \gls{psoc} can be changed during compile time to different values. The lowest internal voltage reference available is 0~-~\SI{2,048}{\volt} range represented with 12 bits, giving a resolution of \SI[per-mode=symbol]{0,50}{\milli\volt\per\bit}. 
To aquire a useful reading it is estimated that at least 12 data points are required to represent the signal swing and thus an amplification of 10 times is wanted. Additionally the signal should be biased to around \SI{1,024}{\volt} DC to make sure all signal components are within the range of the \gls{adc}. 

To amplify the signal a TS462 operational amplifier is used because its low noise and availability during the design. The TS462 comes in packages of two op amps. The first one is used to provide the balanced to unbalanced effect and the second one is used to amplify the signal further. It is chosen not to use a negative power supply and therefore the bias level is kept between the op amps. To let the signal stay within the \SI{5}{\volt} of the power supply the bias is lowered to \SI{1,25}{\volt} at the voltage divider next to R303 on \autoref{fig:mixer_output}. 

The output of the second op amp is AC-coupled and biased to approximately \SI{1}{\volt}. The capacitors on the voltage dividers are used to filter out power supply noise.

The full circuit for the LT5560 is seen in \autoref{appendix:mixerfig}. 


The low frequency signal from the mixer is the signal of interest, but discussed earlier it is seen that a high frequency component is also present on the output. The low frequency component is at a frequency of $\Delta f = \SI{333}{\kilo\hertz}$, and the high frequency component is at a frequency of 
\[ 2f_{RF} + \Delta f = 2 \cdot 869 \cdot 10^{6} + 333 \cdot 10^{3} = \SI{1,971}{\giga\hertz}.\]
While the TS462 does not have a gain above 1 at \SI{2}{\giga\hertz}, a filter is wanted to reduce the high frequency part of the signal further.

\subsection{Design of low-pass filter}\label{sec:designLPFilter}
As seen in \autoref{subsec:downconversion} a filter to remove a unwanted high frequency component from the signal is needed. The filter should have the following characteristics to remove the unwanted frequency components. 
\begin{itemize}
\item Analogue filter since the filter has to filter out frequencies above \SI{1}{\mega\hertz}
\item Low-pass filter 
\item Maximum passband flatness to reduce altering the signal as much as possible.
\item Cut-off frequency at \SI{0.5}{\mega\hertz} (3 dB attenuation point) as described in \autoref{sec:mixercircuitdesign}.
\item An attenuation of 60 dB at \SI{10}{\mega\hertz} since it is desired to filter out any harmonics.
\item Load resistance of \SI{50}{\ohm}
\end{itemize}
With this cut-off characteristic the \SI{333}{\kilo\hertz} information carrying signal, pass through the filter unaltered, while the unwanted  \SI{2}{\giga\hertz} signal is filtered away. A load resistance of \SI{50}{\ohm} is to ensure good signal integrity and power transfer.

A filter with a flat passband is wanted therefore the Butterworth filter type is chosen. To find the necessary order for such a Butterworth filter \autoref{eq:OrderNecessary} is used \citep{AnagogFilters}. 
\begin{equation} \label{eq:OrderNecessary} 
n = \frac{1}{2\cdot \log_{10}\left( \frac{\omega_{s}}{\omega_{p}} \right)} \cdot \log_{10}\left( \frac{10^{ \left( \frac{\alpha_{s}}{10} \right) }-1}{10^{\left( \frac{\alpha_{p}}{10} \right)}-1}  \right) \addunit{1}
\end{equation}
\startexplain
\explain{n is the necessary order}{\si{1}}
\explain{$\omega_{s}$ is the stopband angel}{\si{\radian\per\second}}
\explain{$\omega_{p}$ is passband angel (Cut-off frequency) }{\si{\radian\per\second}}
\explain{$\alpha_{s}$ is the stopband attenuation}{\si{1}}
\explain{$\alpha_{p}$ is the  passband attenuation}{\si{1}}
\stopexplain

The wanted design characteristics are inserted in \autoref{eq:OrderNecessary} and the needed order of the filter is found by rounding up the value. 
\begin{equation} \label{eq:FindOrderNecessary} 
n = \frac{1}{2\cdot \log_{10}\left( \frac{10 \cdot 10^{6} \cdot 2\pi}{0.5 \cdot 10^{6} \cdot 2\pi} \right)} \cdot \log_{10}\left( \frac{10^{ \left( \frac{60}{10} \right) }-1}{10^{\left( \frac{3}{10} \right)}-1}  \right) = 2.307 \to 3
\end{equation}

Due to the high frequency a passive “ladder” filter realisation is chosen. A circuit diagram of the filter is seen on \autoref{fig:LPFilter_realization}. 
\begin{figure}[h]
    \centering
        \includegraphics[width=0.6\textwidth]{figures/circuits/ladder_realization_circuit}
        \caption{Circuit diagram showing a passive third-order “ladder” filter realisation.}
        \label{fig:LPFilter_realization}
\end{figure}

\subsubsection{Finding the transfer function for the circuit}
Looking at \autoref{fig:LPFilter_realization} two \gls{kcl} equations are made, \autoref{eq:LPFilterKCL1} and \autoref{eq:LPFilterKCL2}. 
\begin{equation} \label{eq:LPFilterKCL1} 
\frac{V_{C2} - V_{in}}{sL_{1}} + \frac{V_{C_{2}} - 0}{\frac{1}{sC_{2}}} + \frac{V_{C_{2}} - V_{out}}{sL_{3}} = 0
\end{equation}
\begin{equation} \label{eq:LPFilterKCL2} 
\frac{V_{out} - 0}{R_{L}} + \frac{V_{out} - V_{C_{2}}}{sL_{3}} = 0
\end{equation}

From \autoref{eq:LPFilterKCL1} and \autoref{eq:LPFilterKCL2} a transfer function for the circuit can be deduced, note that $R_{L}$ is normalized to 1.
\begin{equation} \label{eq:LPFilterTfansfer1}  
\frac{V_{out}}{V_{in}} = H(s) = \frac{1}{(L_1 \cdot C_2 \cdot L_3) \cdot s^{3}+(L_1 \cdot C_2) \cdot s^{2}+(L_1+L_3) \cdot s + 1}
\end{equation}

\subsubsection{General third-order butterworth transfer function}
The poles of a butterworth filter can be found with \autoref{eq:LPFilterPoles} \citep{AnagogFilters}. 
\begin{equation} 
p_{k} = e^{j \cdot \left( \frac{2k -1}{2n} \cdot \pi +\frac{\pi}{2} \right) } \label{eq:LPFilterPoles} 
\end{equation}
\startexplain
\explain{$p_{k}$ is a pole}{1}
\explain{$k$ is the pole index $ | k \in \mathbb{N} [ 1 : n]$}{1}
\explain{$n$ is the filter order}{1}
\stopexplain

Since it is a third-order filter, there is three poles, using \autoref{eq:LPFilterPoles} these three poles are calculated. 
%\begin{equation} \label{eq:LPFilterPole1} 
%p_{1} = e^{j \cdot \left( \frac{2 \cdot 1 -1}{2 \cdot 3} \cdot \pi +\frac{\pi}{2} \right) } = e^{j \cdot \frac{2}{3}\pi}
%\end{equation}
%\begin{equation} \label{eq:LPFilterPole2} 
%p_{2} = e^{j \cdot \left( \frac{2 \cdot 2 -1}{2 \cdot 3} \cdot \pi +\frac{\pi}{2} \right) } = -1 
%\end{equation}
%\begin{equation} \label{eq:LPFilterPole3} 
%p_{3} = e^{j \cdot \left( \frac{2 \cdot 3 -1}{2 \cdot 3} \cdot \pi +\frac{\pi}{2} \right) } = e^{j \cdot \frac{4}{3}\pi}
%\end{equation}
\begin{align} \label{eq:LPFilterPolesFound}
 p_{k} &=
  \begin{cases}
   e^{j \cdot \frac{2}{3}\pi}   & \text{for k = 1} \\
   -1							& \text{for k = 2} \\
   e^{j \cdot \frac{4}{3}\pi}   & \text{for k = 3}
  \end{cases}
\end{align}

The transfer function can be made from the three poles found in \autoref{eq:LPFilterPolesFound}.
\begin{subequations}
\begin{align}  
H(s) &= \frac{1}{(s-p_{1})(s-p_{2})(s-p_{3})} \label{eq:TFfromPoles1} \\
 &= \frac{1}{s^{3} + 2s^{2}+2s+1} \label{eq:TFfromPoles5} 
\end{align}
\end{subequations}

\subsubsection{Comparing the two transfer functions to find the normalized values}
Now two different transfer functions for the filter are made. 
By comparing \autoref{eq:LPFilterTfansfer1} and \autoref{eq:TFfromPoles5}, three equations with three unknown variables are made.
\begin{subequations}
\begin{align}
L_1 \cdot C_2 \cdot L_3 &= 1  \label{eq:LPFilter3equations1} \\
L_1 \cdot C_2 &= 2 \label{eq:LPFilter3equations2} \\
L_1 + L_3 &= 2 \label{eq:LPFilter3equations3}
\end{align}
\end{subequations} 

Solving \autoref{eq:LPFilter3equations1}, \autoref{eq:LPFilter3equations2} and \autoref{eq:LPFilter3equations3} for the values of $L_{1}$, $C_{2}$ and $L_{3}$, yields the normalized component values which is seen in \autoref{tab:NormalizedLPFilertValues}. 
\begin{table}[h]
	\centering
	\caption{Normalized component values for a third-order butterworth "ladder" type filter, as the one shown on \autoref{fig:LPFilter_realization}.}\label{tab:NormalizedLPFilertValues}
	\begin{tabular}{l l }
		\textbf{Component}	& \textbf{Value}	\\ \toprule \rowcolor{lightGrey}
		$L_{1n}$	& \SI{1.5}{\henry}		\\ 
		$C_{2n}$	& $\frac{4}{3}$~\si{\farad} 	\\ \rowcolor{lightGrey}
		$L_{3n}$	& \SI{0.5}{\henry} 		\\ 
		$R_{Ln}$	& \SI{1}{\ohm}			\\
	\end{tabular}
\end{table}

\subsubsection{Scaling the normalized values}
The normalized values need to be both frequency and impedance scaled. The frequency scaling factor is given by \autoref{eq:LPFilterFreqScalingFactor} and the impedance scaling factor is given by \autoref{eq:LPFilterImpScalingFactor} \citep{AnagogFilters}. 
\begin{equation} \label{eq:LPFilterFreqScalingFactor}
k_{f} = \frac{\omega_{p}}{\omega_{pn}} = \frac{2\pi \cdot 500 \cdot 10^{3}}{1}= 1 \cdot 10^{6} \cdot \pi
\end{equation}
\begin{equation} \label{eq:LPFilterImpScalingFactor} 
k_{z} = \frac{R_{L}}{R_{n}} = \frac{50}{1} = 50 
\end{equation}

The actual component values for the filter can be found with two scaling equations. To find the inductor values \autoref{eq:LPFilterCApasitorScaling} is used. 
 \begin{equation} \label{eq:LPFilterCApasitorScaling}
L_{is} = \frac{k_{z}}{k_{f}} \cdot L_{in} \addunit{\henry}
\end{equation}
\startexplain
\explain{$i$ is the index number}{\si{1}}
\explain{$L_{is}$ is a scaled inductor}{\si{\henry}}
\explain{$L_{in}$ is a normalized inductor}{\si{\henry}}
\stopexplain
To find the capacitor values \autoref{eq:LPFilter:inductorScaling} is used.
\begin{equation} \label{eq:LPFilter:inductorScaling} 
C_{is} = \frac{1}{k_{f} \cdot k_{z}} \cdot C_{in} \addunit{\farad}
\end{equation}
\startexplain
\explain{$C_{is}$ is a scaled capacitor}{\si{\farad}}
\explain{$C_{in}$ is a normalized capacitor}{\si{\farad}}
\stopexplain

All the needed equations and values have now been found and the actual component values for the filter can be found, by inserting the found normalized component values from \autoref{tab:NormalizedLPFilertValues} together with the scaling factors found in \autoref{eq:LPFilterFreqScalingFactor} and \autoref{eq:LPFilterImpScalingFactor}, into \autoref{eq:LPFilterCApasitorScaling} and \autoref{eq:LPFilter:inductorScaling}. Due to component tolerances the calculated component values, are not easily available for this project, therefore the closest value available is used. 
\begin{subequations}
\begin{align} 
L_{1s} &= \frac{50}{1 \cdot 10^{6} \cdot \pi} \cdot 1.5  = \SI{23.9}{\micro\henry} \xrightarrow{E6} \SI{22}{\micro\henry} \label{eq:LPFilter:inductorScalingDone2} \\
C_{2s} &= \frac{1}{1 \cdot 10^{6} \cdot \pi \cdot 50} \cdot \frac{4}{3} = \SI{8.5}{\nano\farad} \xrightarrow{E6} \SI{8.2}{\nano\farad}  \label{eq:LPFilter:inductorScalingDone1} \\
L_{3s} &= \frac{50}{1 \cdot 10^{6} \cdot \pi} \cdot 0.5  = \SI{8.0}{\micro\henry} \xrightarrow{E6} \SI{8.2}{\micro\henry} \label{eq:LPFilter:inductorScalingDone3} 
\end{align}
\end{subequations} 

\subsubsection{Simulation}
With the circuit designed and all the component values found, the final circuit is simulated in LT-Spice. The result of the simulation is seen on \autoref{fig:LPFilterSImPlot}. 
\begin{figure}[h!]
    \centering
        \includegraphics[width=\textwidth]{figures/design/LPFilter/LPFilterSimPlot}
        \caption{LT-Spice simulation of the circuit on \autoref{fig:LPFilter_realization} with calculated values. The gray boxes represents the limits introduced by the requirements}
        \label{fig:LPFilterSImPlot}
\end{figure}

It is seen from the simulation on \autoref{fig:LPFilterSImPlot} that the passband is flat. It is seen that the attenuation at \SI{5.15}{\mega\hertz} is 60 dB. Furthermore it is seen that the cut-off frequency is located at \SI{543}{\kilo\hertz}. 

The cut-off frequency is of the realised filter is offset by \SI{43}{\kilo\hertz} compared to the design parameter, but the filter has a sharper cut-off curve. It is seen on \autoref{fig:LPFilterSImPlot} that the filter meets the requirements even after a frequency shift. The filter design is accepted. 
%
\subsection{\gls{adc}} \label{sec:adc}
The signal from the mixers is a downconverted version of the signals from the anntennas. These three signals are lowpass filtered so in practice only the low frequency part remains. The signals still contains the phase information and an \gls{adc} is used to sample the three signals. The frequency of these signals should not exceed \SI{543}{\kilo\hertz}, since this is the low-pass filter cutoff frequency and the frequency of the signal should be above \SI{10}{\kilo\hertz}, which is the specified lowest operating range for the chosen mixer \citep{datasheet:LT5560}. Ideally the three signals should be sampled at exactly the same time in order to extract the phase difference, sampled at a high resolution, and at a sampling frequency above $2\cdot\SI{543}{\kilo\hertz} = \SI{1.086}{\mega\hertz}$ because of Nyquist rate.
In the chosen prototyping board for signal processing \gls{psoc} 5LP there are two identical Successive Approximation \glspl{adc} \cite{datasheet:PSoC5LP:_CY8C58LP_Family} refered to as SAR \gls{adc}. This presents an issue for sampling three signals at the same time. The maximum sampling frequency of the SAR \glspl{adc} is \SI{1}{\mega\hertz} at the maximum resolution, \SI{12}{\bit} \citep{datasheet:PSoC5LP:_CY8C58LP_Family}.
It is chosen that sampling at \SI{1}{\mega\hertz} should suffice since simulations and tests show that aliasing introduced by sampling at a frequency that is too low, does not introduce much of an error, unless the signal is sampled at a frequency that is exactly an integer divison of the frequency of the signal. Tests show that the phase approximation from multiplication method using the \gls{psoc} result in a phase approximation up to a signal frequency of \SI{5}{\mega\hertz}. 
Because of the high sampling frequency and parallel data transfer required, a \gls{dma} is used to transfer the data from the \gls{adc} to the memory, which then can be accessed by the program. A \gls{dma} is implemented using a \gls{psoc} hardware block. Each \gls{dma} is set to store the incoming data to a specific address in the microchip.

To sample at exactly the same time, the same external clock is used which runs at \SI{18}{\mega\hertz} as this the highest maximum frequency for the \gls{adc} \citep{datasheet:PSoC5LP:_CY8C58LP_Family}. When using an external clock the maximum conversion rate becomes \SI{0.666}{} \gls{msps} because of hardware limitations of the \gls{adc}. When using a clock frequency above \SI{1}{\mega\hertz} and the internal $V_{ref}$, bypass capacitors have to be placed between the $V_{refs}$ and $GND$ \citep{datasheet:saradc}.

\begin{figure}[h!]
\centering
\includegraphics[width=0.95\linewidth]{ADCTopDesign2}
\caption{Block diagram of configuration of ADCs to achieve sampling of signals from the three antennas at the same time.}
\label{fig:ADCTopDesign2}
\end{figure}

To compensate for having only two \glspl{adc} a multiplexer is used to choose which of two of the three signals to sample at a time. This is possible because only the phase difference between two signals is needed at a time. The two antennas in the horizontal plane provide a phase difference in relation to the angle in the azimuth plane, and the phase difference between two antennas that are placed vertically provide a phase difference in relation to the angle in the elevation plane.
On \autoref{fig:ADCTopDesign2} it is shown how the \gls{adc} is implemented in the \gls{psoc}. When a new set of values is needed from the \gls{adc}, first the analog multiplexer \texttt{ADC_MUX} is set to use the single ended analog input \texttt{PIN_ADC_In_1}. Then the external clock, \texttt{ADC_Clock} start and provide the same clock for both \glspl{adc}. To ensure that each sample occur at the same time the clock \texttt{SOC_Clock} triggers each \gls{adc} to sample. After each DMA has stored 1000 samples, an interrupt is triggered letting the software know that new data is available and the \texttt{ADC_Clock} and \texttt{SOC_Clock} is stopped and the 1000 point array are copied using \texttt{memcpy} to another address in the \gls{psoc}. Before sampling again, any pending \gls{dma} data request is cleared since this proved an issue. \texttt{Pin_ADC_In_2} is then selected, and the process is repeated.

The output is four arrays (two pairs) which can be processed to extract the phase information.

\subsubsection{Quantization error}\label{subsec:quantization_error}
When sampling an analog signal in a finite resolution quantization noise is generated. The noise can be reduced by increasing the sampling resolution or by oversampling. Since this project requires quick batches of 1000 values several times a second the oversampling method is considered unsuitable as it would prolong calculation times. 
The worst possible quantization error of an ideal \gls{adc} is half of the minimum voltage necessary for the \gls{lsb} to change. It implies the possibility of two different analog values having the same digital value. The minimal voltage difference necessary to change the \gls{lsb} is as seen in \autoref{eq:QuantizationError} \citep{AppReport:ADCQuantization}.

\begin{equation} \label{eq:QuantizationError}
q=\frac{A}{2^{n-1}-1}\approx \frac{A}{2^{n-1}} \addunit{\si{\volt}}
\end{equation}
\startexplain
\explain{$q$ is the minimal voltage difference necessary to change the \gls{lsb}}{\si{\volt}}
\explain{$A$ is the amplitude of the converted signal}{\si{\volt}}
\explain{$n$ is the number of bits of the \gls{adc}}{1}
\stopexplain

The quantization error is seen as noise. It is possible to calculate the average power of the noise in respect to the error, and the average power of the signal (in this case a sine wave) by their mean square level as \autoref{eq:TotalQuantizationError} assuming an ideal converter \citep{AppReport:ADCQuantization}.

\begin{subequations} \label{eq:TotalQuantizationError}
	\begin{align}
		&N^2 =\frac{1}{q}\int_{-\frac{q}{2}}^{\frac{q}{2}}E^2dE=\frac{q^2}{12} \\
		&S^2 = \frac{1}{2\pi}\int_{0}^{2\pi}A^2sin^2(\omega t)d\omega t=\frac{A^2}{2}
	\end{align}
\end{subequations}
\startexplain
\explain{$N$ is the \gls{rms} of the quantization noise power}{1}
\explain{$E$ is the error defined as $E=\frac{1}{q}$ for $-\frac{q}{2}<E<\frac{q}{2}$ else 0}{1}
\explain{$S$ is the \gls{rms} of the signal power}{1}
\explain{$\omega$ is the frequency of the signal}{\si{\radian\per\second}}
\stopexplain

From \autoref{eq:TotalQuantizationError} the \gls{snr} can be deduced to get a good idea of the performance of an ideal \gls{adc} depending on its resolution. In this case the \gls{snr} is given by \autoref{eq:SNRADC} \citep{AppReport:ADCQuantization}.
\begin{equation} \label{eq:SNRADC}
		\gls{snr}=10\log_{10}\left( \frac{S^2}{N^2} \right)\approx 6.02n + 1.76 \addunit{\decibel}
\end{equation}
%\todo[author=Mads J]{I cannot make sense of this calculation after reading through these thoroughly: http://www.analog.com/media/en/training-seminars/tutorials/MT-001.pdf
%http://www.ti.com/lit/an/slaa013/slaa013.pdf 
%But the conclusion is correct.}
%

The resolution of the \gls{psoc}'s \gls{adc} can either be \SI{8}{\bit}, \SI{10}{\bit} or \SI{12}{\bit}.
So the \glspl{snr} from quantization for \SI{8}{\bit}, \SI{10}{\bit} and \SI{12}{\bit} \glspl{adc} are respectively \SI{49,9}{\decibel}, \SI{62,0}{\decibel} and \SI{74,0}{\decibel}. 
In \autoref{PhaseDifferenceDetection} in \autoref{eq:phase_reqsnr} it is described that a \gls{snr} > \SI{46,7}{\decibel} is required. Because a significant amount of noise is introduced by other sources in the system, it is chosen to minimize the noise in the \gls{adc}, at a compromise of speed, by using a \SI{12}{\bit} resolution.

In conclusion the three signals are sampled in pairs of two at the same time with a sampling frequency of 0.666 \gls{msps} with a resolution of \SI{12}{\bit}, which introduces a quantization error of approximately \SI{74,0}{\decibel}, when the \gls{adc}'s reference voltage is exactly equal to the maximum voltage of the signal. With the \gls{adc} implemented, an algorithm for determining the phase difference of the sampled signals is made.
%
\section{Calculation of phase} \label{sec:calc_of_phase}
After sampling the signals, they have to be processed to determine the phase difference between the antennas. This project investigates two different methods of determining the phase difference:

\begin{enumerate}
	\item Phase determination by \gls{dft}
	\item Phase determination by multiplication
\end{enumerate}

\subsection{Phase determination by \gls{dft}}

The \gls{dft} is a method that makes a spectrum analysis of a discrete real signal by transforming it into a discrete frequency spectrum as presented in \autoref{eq:DFTTrans}.

\begin{equation}\label{eq:DFTTrans}
S[k] = \sum_{n=0}^{N-1}s[n]\cdot\exp^{-j\cdot \frac{k\cdot n\cdot 2\cdot \pi}{N}}
\end{equation}
\startexplain
\explain{$S[k]$ is the k'th element of the fourier transformed signal}{1}
\explain{$s[n]$ is the n'th element of the sampled signal}{1}
\explain{$N$ is the the number of samples}{1}
\explain{$n$ is the time discrete signal sample number, where $n \in \mathbb{N}$}{1}
\explain{$k$  is the fourier transformed signal sample number, where $n \in \mathbb{N}$}{1}
\stopexplain

Since the output is a complex value the phase difference at a specific frequency can be calculated as in \autoref{eq:PhaseDFT}.

\begin{equation} \label{eq:PhaseDFT}
\varphi [k]= \arctan\left(\frac{Q[k]}{I[k]}\right)
\end{equation}
\startexplain
\explain{$\varphi[k]$ is the $k$'th phase of the signal}{\si{\radian}}
\explain{$Q[k]$ is the $k$'th imaginary part of the signal at the frequency}{1}
\explain{$I[k]$ is the $k$'th real part of the signal at the frequency}{1}
%\explain{$N$ is the the number of samples}{1}
\stopexplain

In \cite{TechReport:DirectionFindingPaper} the author's design a system which determines the determines the \gls{aoa} using the phase difference of three signals. The phase difference is determined using \gls{fft} and has a precision of $\pm$ \SI{1.75}{\milli\radian} using realistic noise models. They find that using a 2048 point \gls{fft} takes \SI{1.2}{\second} because of various latencies in their system. This latency would be too large to fulfil the requirements of this project.  
Although using the \gls{fft} to estimate the phase of the signals is proven to provide a precise estimate of the \gls{aoa}, but considering the limited processing power of the prototyping platform \gls{psoc} and evident time consuming calculations this project aims for a different approach.

\subsection{Phase determination by multiplication}\label{subsec:phaseDiff}
A signal can be defined as in \autoref{eq:SigProcSignalDefinition}. 


\begin{equation}\label{eq:SigProcSignalDefinition}
S_k[n]=A_k \cos \left(\frac{2\pi f_k}{ f_s} n+\phi_k \right)	\addunit{\volt}
\end{equation}
\startexplain
\explain{$S_k[n]$ is the $k$'th signal}{1}
\explain{$A_k$ is the amplitude of the $k$'th signal}{1}
\explain{$\phi_k$ is the phase of the $k$'th signal}{\si{\radian}}
\explain{$f_k$ is the frequency of the $k$'th signal}{\si{\hertz}}
\explain{$f_s$ is the sampling frequency}{\si{\hertz}}
\explain{$n$ the sample number, where $n \in \mathbb{N}$}{1}
\explain{$k$ is the signal number, where $k \in \{ 1,2,3 \} $}{1}
\stopexplain

By multiplication of signals $S_1[n]$ and $S_2[n]$ \autoref{eq:SigProcSignalMultiplication} can be derived.
\begin{align}
S_{12}[n] &= S_1[n]\cdot S_2[n] \\
&= A_1 \cos \left(\frac{2\pi f_1}{ f_s} n+\phi_1 \right) \cdot A_2 \cos \left(\frac{2\pi f_2}{ f_s} n+\phi_2 \right) \addunit{\volt} \label{eq:SigProcSignalMultiplication}
\end{align}

By the trigonometric identities \autoref{eq:SigProcTrigIdent} is deduced.
\begin{equation}\label{eq:SigProcTrigIdent}
S_{12}[n]=\frac{A_1 A_2}{2}\left(  \cos \left(\frac{2\pi n ( f_1 - f_2)}{f_s}+ \phi_1 -\phi_2 \right) + \cos \left(\frac{2\pi n(f_1+f_2)}{ f_s} +\phi_1+\phi_2 \right)  \right) \addunit{\volt}
\end{equation}

$f_1 = f_2$ Since both signals originates from equivalent mixings of the drone's signal and the same oscillator. This results in a \gls{dc} level containing the phase difference information since $f_1 - f_2 = 0$ similarly to the mixing in \autoref{sec:tracking_mixer}. The mean of a cosine is zero and by taking the mean of $S_1[n]\cdot S_2[n]$ the DC level is isolated as seen in \autoref{eq:SigProcLowPass}.

\begin{equation}
	\implies \text{mean} \left( S_{12}[n]\right) \approx \frac{A_1 A_2}{2} \cos \left(  \phi_1-\phi_2 \right) \addunit{\volt}  \label{eq:SigProcLowPass}  \\ 
\end{equation}

The phase-difference is then given by \autoref{eq:SigProcPhaseDiff}.
\begin{equation}\label{eq:SigProcPhaseDiff}
\phi_2-\phi_1=  \arccos \left( \text{mean} \left( S_{12}[n]\right) \cdot \frac{2}{A_1 A_2}\right) \addunit{\radian}	
\end{equation}

However, this method only yields the absolute value of the phase difference. Information about whether the phase difference is negative or positive is needed. Shifting one signal by $\frac{\pi}{2}$ mean that the absolute value of the phase difference is centred around $\frac{\pi}{2}$ and both a positive and a negative phase shift ranging from $\frac{-\pi}{2}$ and $\frac{\pi}{2}$ can be determined.

The only information truly needed in order to use this method is the amplitude of $S_{1}[n]$ and $S_{2}[n]$ to normalize them and so get \autoref{simpArcos}.

\begin{equation}\label{simpArcos}
\phi_2-\phi_1=  \arccos \left( \text{mean} (S_{12}[n]) \cdot2\right) -\frac{\pi}{2} \addunit{\radian}	
\end{equation}

If the signal is assumed to be a sine wave then the amplitude of a signal can be determined by the \gls{rms} method as seen in \autoref{eq:sqrtmean}.

\begin{subequations} \label{eq:sqrtmean}
	\begin{align}
	&A_1 = \sqrt{\frac{\sum\limits_{n=0}^{x-1}{S_{1}[n]}}{x}}\cdot\sqrt{2} \\
	&A_2 = \sqrt{\frac{\sum\limits_{n=0}^{x-1}{S_{2}[n]}}{x}}\cdot\sqrt{2}
	\end{align}
\end{subequations}
\startexplain
\explain{$x$ is the number of samples}{1}
\stopexplain

If the signal is a sine or cosine wave then the amplitude can be determined exact. However it is noise sensitive but as mentioned earlier the \gls{snr} should be above \SI{46.7}{\decibel} as mentioned in \autoref{PhaseDifferenceDetection}. A simulation in MATLAB with 1000 points which is in \autoref{fig:PhaseDiffSimulation} shows that using the multiplication method produce an average error of \SI{1.4}{\milli\radian}. Tests using two signal generators with a programmable phase difference, and the \gls{adc} described above an average precision of \SI{1.89}{\milli\radian} was achieved. Precision better than \SI{2.08}{\milli\radian} is achievable and tests of the multiplication method (in \autoref{appendix:aoaestimationerr}) proves it is fast and precise.

\begin{figure}[h!]
	\centering
	\includegraphics[width=\textwidth]{figures/design/UglyFigure}
	\caption{Simulation of the phase difference program with the multiplication signal method. The signal is applied white Gaussian noise to have a \gls{snr} of \SI{46.7}{\deci\bel}}
	\label{fig:PhaseDiffSimulation}
\end{figure}


%


\section{Conclusion}
The section implements the \gls{aoa} determination using the phase difference of a beacon signal. The frequency has been chosen to \SI{864}{\mega\hertz} and an oscillator for the frequency is designed, which is free running and proves to be very temperature dependent. Three patch antennas for receiving the beacon signal are designed which prove to have a higher efficiency at \SI{869}{\mega\hertz} so the frequency is changed. The transmitter consists of an \SI{869}{\mega\hertz} oscillator, an amplifier and an omnidirectional whip antenna. The amplification is determined from the link budget so the transmission power becomes \SI{3.02}{\deci\belm}. The three received signals are downconverted to a lower frequency below \SI{500}{\kilo\hertz} using a mixer and a local oscillator, and then sampled by an \gls{adc} on a \gls{psoc} which calculates the \gls{aoa} from the phase differences of the received signals with an average error of \SI{1.4}{\milli\radian}.