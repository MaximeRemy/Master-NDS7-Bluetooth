\subsection{Beacon signal}
\todo[author=Jimmy, inline]{Since the different methods have different requirements to the beacon signal, maybe you need to describe what kind of signal is needed for each, instead of a general discussion? (Alexander: I guess this means writing a small description in every section above. I agree, this subsection should be written into the above)}

It was found in the \autoref{sec:TrackingTheDrone} that the majority of methods which can be used to detection and track the drones position all rely on having the drone emitting a beacon signal. 

In this section this beacon signal will be analysed, in order to fully understand the process of tracking a drone. 

A beacon is an object that intentionally attracts attention to a specific location. In this case the beacon attracts attention to the drones location, by emitting a beacon signal made by a electromagnetic wave. 

The frequency and characteristics of the emitted beacon signal, have  to match the the design of the receive or vice versa. 

The beacon signal needed for the time of arrival method, which is explained in \autoref{TimeOfArrival}, should be an impulse. The beacon should emit an impulse whenever the position of the drone is needed. If the drone moves a new impulse must be emitted in order for the ground station to know the new location of the drone. 

With both the signal strength difference and the phase difference detection methods, explained in \autoref{SignalStrengthDifference} and \autoref{PhaseDifferenceDetection} respectfully, the beacon needs to emit a beacon signal consisting of a continuous magnetic wave. 

The needed beacon signals can be made in many different ways,  
dependent on the wanted frequency. If a specific frequency is wanted a specificity designed transmitter circuit is needed. 
If a normalised frequency is sufficient a general purpose transmitter might be used, for example, if the desired frequency is 2,4 Ghz, a WI-Fi card or a Bluetooth module be able to emit the wanted beacon signal. 