\subsection{Conclusion on tracking metods}\label{sec:pre:trackingConclusion}
With all the different methods of tracking the drone investigated a conclusion is to be made. An overview of the conclusion of the different methods are seen in \autoref{tab:TrackingConclusion}.

Optical tracking is discarded as it is too dependent on environmental conditions to be effective.

Using radar technology to track different objects is already a widely used technique however it requires antennas with a high gain or high transmission power levels. The radar method is assessed to be too complex for this project.  

Despite of the inherent simplicity of the signal strength method it does not have the precision required.

Using the on-board \gls{gps} module on the drone to track it is a possible solution had the \gls{gps} been more precise. The \gls{gps} method might however be useful to generate rough estimate of the drones position if it is found needed. 

The time of arrival method shows great tracking potential, but to reach the wanted precision it becomes infeasible due to the need of high speed sampling.

The phase difference method shows many of the same characteristics as the time of arrival method, but without the need for high speed sampling.

\begin{table}[h!]
\centering
\caption{Summary of the different methods of tracking the drones direction.}
\begin{tabularx}{\textwidth}{l X }
	\textit{Method} 	& \textit{Summery} 	\\ \toprule \rowcolor{lightGrey}
	Optical tracking	& Too weather dependent 					\\
	Time of arrival		& Requires a high sampling frequency and sensitive electronics \\ \rowcolor{lightGrey}
	Phase difference 	& Good solution	\\ 
	Signal strength 	& Very noise prone and not precise enough \\\rowcolor{lightGrey}
	GPS					& Easy implemented, not very precise \\ 
	Radar & Complex, require high antenna gain\\
\end{tabularx}
\end{table} \label{tab:TrackingConclusion}

It is chosen to utilize the phase difference method to track the drone. The motorised antenna stand is now analyzed.

