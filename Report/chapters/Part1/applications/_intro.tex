\graphicspath{{figures/Part1/applications/}}

\chapter{\glsentrylong{wpt}} \label{ch:WPTPreanalysis}
The method of \gls{wpt} has many applications. Numerous power transferring technologies exist eg.~inductive coupling, capacitive coupling and power transmission via radiowaves or concentrated lasers \citep{book:shinohara}. The first two are effective in short range. However to expand the applications of wireless power transfer to longer distances, radiowaves and concentrated lasers are more efficient \citep{book:shinohara}.

Long distance \gls{wpt} can reduce the need for high battery capacity in many applications due to the possibility to recharge them periodically.
QI and Qualcom are two companies who already have implemented wireless charging over short distances, mostly for small consumer electronics such as mobile phones, tablets and wearable technology \citep{web:QI} \citep{web:Qualcom}. 

However the \gls{wpt} application could be expanded by using a directional radio wave beam \citep{book:shinohara}. Some of its possible applications could be to power a target from afar or to transfer solar power to earth from a satellite \citep{sci_art:andrea}. 

This project begins by analyzing different methods of \gls{wpt} to determine their feasibility when transferring power wirelessly over long distances. A range of different applications for long distance \gls{wpt} is examined afterwards to identify use cases as well as challenges or problems in implementing them for long distance.

\section{Methods of \glsentryshort{wpt}}\label{sec:WPTmethods}
Different methods of \gls{wpt} are studied in order to determine which methods are feasible for long distances.

All the researched \gls{wpt} methods use a similar setup with a transmitting station transferrring power to a receptor via a wireless medium. Four different methods are studied.
 
\subsection{Inductive coupling} \label{sec:InductiveCoupling}
Inductive coupling consists of two electrically conductive coils placed adjacent to each other at a fixed distance. 

Inductive coupling works by creating a fast changing magnetic field around the power transmitting coil, it is achieved by applying an alternating current to the coil. 
If the power receiving coil is placed within the changing magnetic field, an electric current will be inducted in the power receiving coil, thus transferring electrical energy wirelessly from one coil to the other coil. 
%In order to have the highest efficiency the power receiving coil have to be placed where the magnetic field is strongest.  

An illustration of inductive coupling setup is seen on \autoref{fig:InductiveCouplingSetup}. 
\begin{figure} [h]
	\centering
	\includegraphics[width=0.8\linewidth]{figures/Part1/InductiveCoupling}
	\caption{Setup demonstrating \gls{wpt} using inductive coupling \citep{website:inductive_coupling}.}
	\label{fig:InductiveCouplingSetup}
\end{figure}

Inductive coupling is shown to have a very high efficiency at short range, which dramatically falls when the range is increased \citep{website:Analysis_of_Wireless_Power_Transmission}. To expand the range of the inductive coupling, resonant circuits can be introduced to carry the magnetic field further. This is called resonant coupling.
%
%\todo[author=Daniel, inline]{ Source on topic: https://arxiv.org/ftp/arxiv/papers/1311/1311.5382.pdf}
%
%\todo[author=Daniel, inline]{Source on topic: http://www.jpier.org/PIERM/pierm34/20.13121908.pdf}
%
\subsubsection{Resonant coupling}\label{sec:ResonantCoupling}
Resonant coupling is an expansion on the inductive coupling. It consists of an inductive coupling with one or more resonating circuits introduced into the changing magnetic field.

An illustration of resonant coupling setup with two supplementing coils in between is seen on \autoref{fig:ResonantCouplingSetup}. 
\begin{figure} [h]
	\centering
	\includegraphics[width=0.8\linewidth]{figures/Part1/ResonantInductiveCoupling}
	\caption{Resonant coupling setup with two supplementing coils between \citep{website:resonant_inductive_coupling}.}
	\label{fig:ResonantCouplingSetup}
\end{figure}

Resonant coupling can have very high efficiency at medium range, but like with inductive coupling it exponentially decreases when the range is increased, the addition of more coils will however improve efficiency and allow for a longer transferring range.

A team of MIT students showed that it is possible to transfer energy over a distance of 2 meters, with an efficiency of 40\%, by using this method \citep{website:WiTricity_Highly_Resonant_Wireless_Power}. Subsequently Intel reproduced the setup and shown an efficiency of 75\% at a shorter distance \citep{TechReport:WPTSheik}.

\subsection{Capacitive coupling}\label{sec:capacitiveCoupling}
The power is transferred by having four conducting plates, two at the power transmitting part and two at the power receiving part, placed in pairs of two adjacent to one and other.

The capacitive coupling makes use of the E-field in order to transfer power wirelessly. The method utilizes the principle of a capacitor using the E-field to cause a current to run between the receivers to plates, and thereby transfer power wirelessly.

On \autoref{fig:capacitiveCoupling} a illustration of a capacitive coupling setup is seen \citep{sci_article:capacitively_coupled_power_transfer}.
\begin{figure}[h]
	\centering
	\includegraphics[width=0.6\linewidth]{figures/Part1/CapacitiveCouplingPDF}
	\caption{Capacitive Coupling setup \citep{website:capacitive_bipolar}.}
	\label{fig:capacitiveCoupling}
\end{figure}

This method relies on plates being close enough to each other to behave as capacitors. Therefore this method is not suitable for energy transfer over longer distances.
%\todo[author=Daniel, inline]{Source on topic: http://ieeexplore.ieee.org/stamp/stamp.jsp?arnumber=6566413 }

%The Capacitive coupling setup can be modelled as a circuit, where the pairs of power transmuting plates can be considered as capacitors. 
%
%circuit diagram here. 
%
%writing up the voltage division for the circuit yields: 
%\begin{equation}
%V_{out} = \frac{R}{R + \frac{1}{s \cdot C_1 } + \frac{1}{s \cdot C_2 }}  \cdot V_{in} 
%\end{equation}
%\begin{equation}
%V_{out} = \left| \frac{R}{R + \frac{1}{j \cdot \omega \cdot C_1 } + \frac{1}{j \cdot \omega \cdot C_2 }}  \cdot V_{in} \right|
%\end{equation}
%In order to have the highest possible output voltage, the capacitance of $C_1$ and $C_2$ should have as high as possible. The frequency should be as high as possible. Furthermore the load, R should be very big compared to $\frac{1}{s \cdot C_1,2}$. 
%
%The size of C_1 and C_2 is determined by the physical dimensions of the  two pairs of plates. 
%\begin{equation}
%C = \frac{\varepsilon \cdot A}{d}
%\end{equation}
%In order to have a high capacity A should be larger then d.  
%
%Assuming the four plates is going to be equal in size, the equations combined yields: 
%\begin{equation}
%V_{out} = \left| \frac{R}{R + \frac{1}{j \cdot \omega \cdot \frac{\varepsilon \cdot A}{d} } + \frac{1}{j \cdot \omega \cdot \frac{\varepsilon \cdot A}{d} }}  \cdot V_{in} \right|
%\end{equation}
%
%isolation for A 
%\begin{equation}
%A = \left| \frac{2 \cdot d \cdot V_{out}}{R \cdot \varepsilon \cdot j \cdot \omega \cdot (V_{in} - V_{out}) } \right|
%\end{equation}
%\begin{equation}
%A = \left| \frac{2 \cdot 20 \cdot 50}{R \cdot \varepsilon \cdot j \cdot \omega \cdot (V_{in} - V_{out}) } \right|
%\end{equation}








\subsection{Photovoltaic power transfer}\label{sec:PhotovoltaicPowerTransfer}
The photovoltaic power transfer makes use of light to transfer power wirelessly. The power sending station transforms electrical energy into light, which is directed in form of a beam to the receiving station.

At the power receiving station the energy stored in the light beam is obtained and transformed back to electrical energy using a photovoltaics cell (solar panel).

The efficiency of the power transfer is mostly determined by the efficiency in conversion to a light beam and the photovoltaic cells. However current photovoltaic cells are not very efficient. \SI{34.5}{\percent} efficiency is considered as one of highest efficiency available on the market \citep{News:most_efficient_solar_cells_ever}.% (Source: http://www.sciencealert.com/engineers-just-created-the-most-efficient-solar-cells-ever)\todo[author=Daniel, inline]{ Add to bib}.

Photovoltaic power transfer is found to be very impractical. For the method to work, the power sending station is to hit the power receiving station with a light beam. If an object e.g a bird or the skies block the light beam from reaching its destination, the transfer fails. 

%Source on topic: http://citeseerx.ist.psu.edu/viewdoc/download?doi=10.1.1.217.9862&rep=rep1&type=pdf


\subsection{Radio waves} \label{sec:radioRadiowaves} 
Radio waves can be used to transfer power wirelessly between antennas.

A frequency converter is used to convert an \gls{ac} signal to the transmitting frequency.
The receiving antenna uses an other frequency converter to convert the high frequency signal to the desired frequency. The two antennas have to point at each other and requires a line of sight in order to achieve maximum transferring efficiency. 
 
The setup for this method is shown in \autoref{fig:RadioAntenna}. 
\begin{figure}[h]
	\centering
	\includegraphics[width=1\linewidth]{figures/Part1/RadioAntenna}
	\caption{Setup with transmitting and receiving antenna of \gls{wpt} using radiowaves.}
	\label{fig:RadioAntenna}
\end{figure}

Radio wave transmission has been found to be very efficient at long distance \citep{TechReport:WPTSheik}.% It is particularly true at \SI{2.45}{\si{\giga\hertz}} which has an efficiency of more than \SI{90}{\percent} \citep{TechReport:WPTSheik}. For power transferring a slotted waveguide antenna type is ideal, because of its high power handling capability which has an efficiency of more than \SI{95}{\percent} \citep{TechReport:WPTSheik}.

In 1987 the Canadian Communications Research Centre (CCRC) researched the idea of having a \gls{wpt} unmanned lightweight airplane act as a relay station for broadcasting services called SHARP. The airplane should circle at an altitude of \SI{21}{\kilo\meter} and stay up months at a time without landing. They successfully constructed a prototype model airplane which flew for 20 minutes at an altitude of \SI{150}{\meter} using a \SI{2.45}{\giga\hertz}, \SI{10}{\kilo\watt} microwave signal \citep{book:shinohara,TechReport:SHARP88}.

The advantage of \gls{wpt} with antennas is the distance for power transfer and the efficiency of the method compared to the other mentioned methods. It also presents a disadvantage that a transmitting antenna can be potentially more dangerous for living beings \citep{manual:ICNIRP_GUIDELINES}, and also needs to have the antennas to be aligned.



%\todo[author=Daniel, inline]{Source for "Reference levels for general public exposure to time-varying electric and magnetic fields": http://www.icnirp.org/cms/upload/publications/ICNIRPemfgdl.pdf}




\subsection{Health and safety concerns}
Magnetic fields can be harmful, at frequencies between 1 - \SI{300}{\hertz} the exposure guideline for a whole human body, states that the exposure should not exceed \SI{60}{\milli\tesla\per f}. In the frequency spectrum from \SI{300}{\hertz} - \SI{30}{\kilo\hertz} the whole or partial body exposure should not exceed \SI{0.2}{\milli\tesla}. For users of pacemakers the limit is \SI{0.1}{\milli\tesla} for frequencies of \SI{60}{\hertz} \cite{web:MagneticFieldSafty}.

The E-field made by some \gls{wpt} methods follows many of the same safety concerns as the once for the Magnetic fields.

Photovoltaic power in a concentrated beam such as a laser can be dangerous for living beings. A \SI{1}{\milli\watt} laser can cause temporary eye damage to a person. If a person is hit directly in the eye with a 10 to \SI{20}{\milli\watt} laser, the person's blink reflex might not be fast enough to avoid eye damage. If the power of the laser is increased further, a burn risk is also introduced. Lasers outputting beams with powers above \SI{250}{\milli\watt} could cause skin burns similar to hot wax \cite{web:LaserSafty}.

Electromagnetic waves can also be dangerous. If biological material is placed in a electromagnetic wave it will experience a dielectric heating. This means that if a person touches or stands around an antenna while a high amount of power is transmitted, this person could get severe burns.

\subsection{Conclusion on \gls{wpt} methods}\label{sec:ConclusionWPT}
On the safety concerns \gls{wpt} methods based on fields are safer than radiations ones as the potentials risks are more manageable and localized.

Inductive coupling is dependent on having two conductive coils being close to one and other in order to reach high efficiency. Resonant coupling is more efficient than the inductive coupling, but is still dependent on having two or more conductive coils being close to one and other. Both inductive coupling and resonant coupling are infeasible for the altitude at which an aircraft operates. 

Capacitive coupling suffers from the same distance scaling problems as inductive and resonant coupling. 

Photovoltaic power transfer is found to be impractical and to have lower efficiency than the alternatives.

\gls{wpt} via radio waves is shown to be feasible over long distances and has a potential to be highly efficient.

In table \ref{tab:mcp_cmd_expl}, a summary of the different methods of \gls{wpt} is present.
\begin{table}[h]
	\centering
	\caption{Summary of the different methods of wireless power transfer.}

	\begin{tabularx}{\textwidth}{lXX}
		Method 				& Summary 										\\ \toprule \rowcolor{lightGrey}
		Inductive coupling	& Only high efficiency at short range. 	\\
		Resonant coupling	& Only high efficiency at medium range with additional coils	\\ \rowcolor{lightGrey}
		Capacitive coupling 		& Only effective at short range. \\
		Photovoltaic 		& Long range, medium-low efficiency, impractical \\\rowcolor{lightGrey}
		Microwave			& Long range, high efficiency \\ 
	\end{tabularx}
\end{table}\label{tab:mcp_cmd_expl}

Thus the conclusion of this section is that radio wave \gls{wpt} is the most suitable method at long distance. The photovoltaic method would be feasible over long distances as well, but with a lower efficiency. 

For both of these method to work a high precision is needed when pointing the power transmitter at the drone. In order to achieve this the ground station needs to locate and track the target with a high accuracy and points very precisely at the target. 

With the methods of wireless power transfer researched, the applications of long distance \gls{wpt} will now be analysed. 


\section{Applications of long distance \glsentryshort{wpt}}
Applications of long distance \gls{wpt} ranges from beaming several \si{\giga\watt} of microwave power from solar power satellites to Earth over a distance of several thousands of \si{\kilo\meter}, to  applications which powers drones flying at an altitude of approximately 20-\SI{50}{\meter}.

This project will examine three different applications which either do benefit or could benefit from long distance \gls{wpt}. The three applications are \glspl{haps}, solar power satellites and drones.

\newpage 
\subsection{\glsentryshort{haps}}
A \gls{haps} is a hovering station which operate at a fixed location at an altitude of 20 - 50 km \citep{manual:radio_regulation}.

The first \gls{haps}' were developed for military use for two specific tasks, the first one being field surveillance and the second one acting as a relay for wireless communication. 
Two publicly known \gls{haps} projects from USA are the Lockheed Martin High Altitude Airship and the project HiSentinel. The ambition of the military is to get services akin to a satellite for a lower operational cost \citep{article:Different_models_HAPS}.

However, \gls{haps} platforms could also fill some roles for the commercial space agencies in the near future. The constantly increasing need for high-speed wireless communications and communication in remote areas make \gls{haps} a good area of study. 
Thanks to its privileged position at an altitude of 17 to \SI{22}{km} it will be able to ignore any geographical restriction imposed by the region, centralize the network and allow better repartition of the resources.
Furthermore the cost of a \gls{haps} could potentially be less than a similar ground station, as a suitable site does not need to be acquired first and a need of relays to spread the network everywhere in the region may be avoided depending on the range of the \gls{haps} \citep{sci_article:haps_for_wireless_comm}.

One of the largest problems with current \gls{haps} technology is power. \gls{haps} as network stations have three main sources of power consumption; engines, payload and maintenance. The Lindstrand balloon requires \SI{90}{\kilo\watt} for propeller and an additional \SI{15}{\kilo\watt} is needed to power a communication station payload \citep{TechReport:HAPS_power}. During the day the solar energy is enough to entirely power the HAPS \citep{sci_article:haps_solar_power}.
However at night solar power is not available forcing the HAPS to get power elsewhere. Current \gls{haps} projects use batteries for overnight power because the distance between the \gls{haps} and the ground is too large for any wired powering.
This presents a problem. The batteries for powering a \gls{haps} for a full night are large and heavy. Using data from \cite{TechReport:HAPS_power} a quick estimate of the battery size can be made.

By considering an average night to be 12 hours, then a \gls{haps} has to be able to store at least $\SI{105}{\kilo\watt} \cdot \SI{12}{\hour} = \SI{1260}{\kilo\watt\hour}$ in its batteries. A typical battery has an energy density of around \SI{300}{\watt\hour\per\kilogram} \citep{report:IEA_batteries}.

The needed mass of batteries needed to keep the \gls{haps} operational doing the night is calculated in equation \ref{eq:HAPS_battery_weight}. 
\begin{equation} \label{eq:HAPS_battery_weight}
	\frac{\SI{1260}{\kilo\watt\hour}}{\SI{300}{\watt\hour\per\kilogram}} = \SI{4200}{\kilogram}
\end{equation}

By removing the batteries alone, without considering housing or power conversion circuitry, \autoref{eq:HAPS_battery_weight} show a \gls{haps} could save \SI{4200}{\kilogram} of mass.

This saving makes \gls{haps} an obvious target for wireless power transfer.
%
\subsection{Solar power satellites}\label{sec:solarPowerSatellites}
Another application of long distance \gls{wpt} is as a part of solar power satellites. A solar power satellite would consist of three parts: solar panels for converting solar energy to DC electricity, energy conversion circuitry depending on choice of \gls{wpt} technology and an implementation of a \gls{wpt} technology for transferring energy to Earth.

The advantages of solar power satellites as opposed to ground stationed solar panels includes a $33\%$ increased maximum intensity of sunlight \citep{website:Light_intensity_at_different_lattitudes}, no interference from weather, potentially higher solar panel efficiency due to lower ambient temperature and approximately 24 hour sun availability. The advantages adds up to approximately eight times the sun exposure which coupled with increased solar panel efficiency, makes solar power satellites an interesting option as a future CO$_2$-free energy source \citep{book:shinohara}.

Two reference models for solar power satellites placed in geostationary orbit have been designed, one by \gls{nasa} in 1978 and one by \gls{jaxa} in 2004. Both of these designs implement microwave antennas to transfer power from the solar power satellite to Earth. Both design use microwave antennas with a diameters of above 1 km. The sheer size of the antennas both in space and on ground makes the design and construction of the microwave power transmission system a big task. Other solar power satellites have been designed to function at lower altitudes, such as the SunTower designed by \gls{nasa} which also utilize microwave power transfer \citep{sci_article:suntower}. High microwave beam collection and transmission antenna efficiencies must be met to make the solar power satellites feasible as a power source. Furthermore accurate tracking and antenna direction control, dependent on the directivity and size of antennas is necessary \citep{book:shinohara}.

For a solar power satellite orbiting in a geostationary orbit, a slight error in the angle when aligning the satellites, will diminish the efficiency significantly. 
An error in angle of $\SI{0,0016}{\degree}$ at a distance of $\SI{36000}{\kilo\metre}$ will result 
in the microwave beam being $\SI{1}{\kilo\metre}$ off target. 
\cite{book:shinohara} approximates that the angle must be kept within a tolerance of $\SI{0,0005}{\degree}$ for a solar power satellite system.
%
\subsection{Drones} \label{sec:app_drones}
A drone is an \gls{uas} also known as \gls{uav}. These vary widely in cost, size and endurance. Drones can be controlled either by a remote control or they can be automated beforehand by programming an embedded microprocessor \citep{TechReport:UAV}. Drones are mostly equipped with small components like a camera but can also be equipped with big components such as weapons \citep{Web:Mili}.

Drones are heavily used nowadays by both military and civilians, and can be bought commercially \citep{Web:Drone}. The military use drones for surveillance and dangerous missions, while civilians use them for aerial photography, monitoring and fun. Civilians' drone have less features because of the size, specification and restrictions. Civilian drones are typically much smaller. Most common marketed drones are under \SI{25}{\kilogram} and have a typical flying time of 20 minutes on a single charge. Given that the technology for wireless charging is developing more and more, a solution for longer flying times would be highly sought.

The reason for the low flying time is because of the battery capacity. The more battery capacity the more the drones will weight. Therefore the designer and manufacturer made a decision to balance between weight and battery capacity. \gls{wpt} could provide a drone a longer flying time or a continuous charging, so the drone would be able to fly for a prolonged period of time.

A ground based charging station could be a solution to supply drones with power wirelessly. The station should track the position and altitude of the drone and point the \gls{wpt} transmitter towards the drone in order to transfer power efficiently. It will be a challenge because of the constant movement of the drone while in the air. %A maximum altitude for the commercial drone called 3DR Solo is \SI{120}{\meter} while the size of the drone is \SI{46}{\centi\meter} x \SI{46}{\centi\meter}. It will be a challenge to point exactly at a moving drone with a distance of 120m \citep{Web:DroneSpec}. 

%
\subsection{Conclusion on applications of long distance \glsentryshort{wpt}}
Long distance \gls{wpt} offers many advantages, such as allowing transmission of sun energy harvested at higher efficiency by satellites in orbit of Earth, and to power drones in applications where longer flight times are required e.g. surveillance. Long distance \gls{wpt} can also be used to power \glspl{haps} while solar energy is unavailable. \glspl{haps} provides a way to deploy communication and surveillance networks in regions where ground based networks are not possible or available. A pre-built \gls{haps} and ground station could be a relatively quick way of reinstating a communications network in a region after a natural catastrophe.

Common for all the different applications of long distance \gls{wpt} is the need for precise tracking and pointing of antennas or lasers to ensure high efficiency when transferring energy. In some applications the transmitter and receiver will move relative to each other e.g. when charging a moving surveillance drone. For applications like these the pointing and tracking systems must both be quick and precise.

%