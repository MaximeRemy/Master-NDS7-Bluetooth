\subsection{Health and safety concerns regarding \gls{wpt}}

The different methods of \gls{wpt} has been research in \autoref{sec:WPTmethods}, all the different methods have some health and safety concerns which needs to be addressed before a product can de designed and constructed. 

The Inductive coupling (\autoref{sec:InductiveCoupling}) and Resonant coupling (\autoref{sec:ResonantCoupling}) methods both utilizes magnetic fields to transmit energy. Magnetic fields can be dangerous, at frequencies between 1 - \SI{300}{\hertz} the exposure guideline for a whole human body, says that the exposure should not exceed \SI{60}{milli\tesla\per f}. In the frequency spectrum from \SI{300}{\hertz} - \SI{30}{\kilo\hertz} the whole or partial body exposure should not exceed \SI{0.2}{\milli\tesla}. For pacemeaker users the limit is \SI{0.1}{\milli\tesla} for frequencies of \SI{60}{\hertz}.  \cite{web:MagneticFieldSafty} 

The capacitive coupling (\autoref{sec:capacitiveCoupling}) method transmits the power by creating an E-field. The E-field made by the capacitive coupling methods follows many of the same safety concerns as the once for the Magnetic fields. 

The Photovoltaic power transfer (\autoref{sec:PhotovoltaicPowerTransfer}) method uses a concentrated laser bean to transferring energy. Lasers can be dangerous for living beings, a \SI{1}{\milli\watt} laser can cause tempearely eye damage to a person, if a person is hit directly in the eye with a 10 - \SI{20}{\milli\watt} laser the persons “blink reflex” might not be fast enough to avoid eye damage. If the power of the laser is increased further, a burn risk is also intrudused, lasers outputting beams with powers above \SI{250}{\milli\watt} could cause skin burns similar to hot wax. \cite{web:LaserSafty}

The transmission of energy using radiowaves (\autoref{sec:radioRadiowaves}) uses electromagnetic waves to transmit its energy. Electromagnetic waves can be dangerous, if some biological material is placed in a electromagnetic wave it will experience a dielectric heating, this means that is a person touches or stands around an antenna while it is transmitter high amount of power, the person could get severe burns. The electromagnetic waves used for this method is equivalent to the once used in microwave ovens. \cite{web:RadioWaveSafty}

With all the analyses of the \gls{wpt} methods and their concerns regarding safety researched a conclusion can be made. 
