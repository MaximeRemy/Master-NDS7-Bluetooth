\subsection{Solar power satellites}\label{sec:solarPowerSatellites}
Another application of long distance \gls{wpt} is as a part of solar power satellites. A solar power satellite would consist of three parts: solar panels for converting solar energy to DC electricity, energy conversion circuitry depending on choice of \gls{wpt} technology and an implementation of a \gls{wpt} technology for transferring energy to Earth.

The advantages of solar power satellites as opposed to ground stationed solar panels includes a $33\%$ increased maximum intensity of sunlight \citep{website:Light_intensity_at_different_lattitudes}, no interference from weather, potentially higher solar panel efficiency due to lower ambient temperature and approximately 24 hour sun availability. The advantages adds up to approximately eight times the sun exposure which coupled with increased solar panel efficiency, makes solar power satellites an interesting option as a future CO$_2$-free energy source \citep{book:shinohara}.

Two reference models for solar power satellites placed in geostationary orbit have been designed, one by \gls{nasa} in 1978 and one by \gls{jaxa} in 2004. Both of these designs implement microwave antennas to transfer power from the solar power satellite to Earth. Both design use microwave antennas with a diameters of above 1 km. The sheer size of the antennas both in space and on ground makes the design and construction of the microwave power transmission system a big task. Other solar power satellites have been designed to function at lower altitudes, such as the SunTower designed by \gls{nasa} which also utilize microwave power transfer \citep{sci_article:suntower}. High microwave beam collection and transmission antenna efficiencies must be met to make the solar power satellites feasible as a power source. Furthermore accurate tracking and antenna direction control, dependent on the directivity and size of antennas is necessary \citep{book:shinohara}.

For a solar power satellite orbiting in a geostationary orbit, a slight error in the angle when aligning the satellites, will diminish the efficiency significantly. 
An error in angle of $\SI{0,0016}{\degree}$ at a distance of $\SI{36000}{\kilo\metre}$ will result 
in the microwave beam being $\SI{1}{\kilo\metre}$ off target. 
\cite{book:shinohara} approximates that the angle must be kept within a tolerance of $\SI{0,0005}{\degree}$ for a solar power satellite system.