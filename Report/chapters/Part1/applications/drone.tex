\subsection{Drones} \label{sec:app_drones}
A drone is an \gls{uas} also known as \gls{uav}. These vary widely in cost, size and endurance. Drones can be controlled either by a remote control or they can be automated beforehand by programming an embedded microprocessor \citep{TechReport:UAV}. Drones are mostly equipped with small components like a camera but can also be equipped with big components such as weapons \citep{Web:Mili}.

Drones are heavily used nowadays by both military and civilians, and can be bought commercially \citep{Web:Drone}. The military use drones for surveillance and dangerous missions, while civilians use them for aerial photography, monitoring and fun. Civilians' drone have less features because of the size, specification and restrictions. Civilian drones are typically much smaller. Most common marketed drones are under \SI{25}{\kilogram} and have a typical flying time of 20 minutes on a single charge. Given that the technology for wireless charging is developing more and more, a solution for longer flying times would be highly sought.

The reason for the low flying time is because of the battery capacity. The more battery capacity the more the drones will weight. Therefore the designer and manufacturer made a decision to balance between weight and battery capacity. \gls{wpt} could provide a drone a longer flying time or a continuous charging, so the drone would be able to fly for a prolonged period of time.

A ground based charging station could be a solution to supply drones with power wirelessly. The station should track the position and altitude of the drone and point the \gls{wpt} transmitter towards the drone in order to transfer power efficiently. It will be a challenge because of the constant movement of the drone while in the air. %A maximum altitude for the commercial drone called 3DR Solo is \SI{120}{\meter} while the size of the drone is \SI{46}{\centi\meter} x \SI{46}{\centi\meter}. It will be a challenge to point exactly at a moving drone with a distance of 120m \citep{Web:DroneSpec}. 
