\section{\glsentryshort{haps}}
\gls{haps} is defined in \cite{manual:radio_regulation} Radio Regulations No. S1.66A as “A station located on an object
at an altitude of 20 – \SI{50}{km} and at a specified, nominal fixed
point relative to the Earth”.
The first HAPS developed were for military use for two specific tasks, the first one being field surveillance and the second one acting as a relay for wireless communication. 
Two publicly known \gls{haps} projects from USA are the Lockheed Martin High Altitude Airship and the project HiSentinel. The ambition of military is to get services akin to a satellite for a lower operational cost.
\todo[author=Alexander]{Add source with different HAPS models}

However \gls{haps} platforms could also fill some roles for the commercial space in the near future. The constantly increasing need for high-speed wireless communications and communication in remote areas make \gls{haps} a good area of study. 
Thanks to its privileged position at an altitude of 17 to \SI{22}{km} it will be able to ignore any geographical restriction imposed by the region, centralize the network and allow better repartition of the resources. 
Furthermore its cost can  be potentially less than a ground station, as a suitable site does not need to be acquired first and a need of relays to spread the network everywhere in the region may be avoided depending on the range of the \gls{haps}.
\todo[author=Alexander]{Add the researchgate source.}

\begin{figure} [h]
	\centering
	\missingfigure{Figure showing the altitude, range or something else about HAPS}
	\caption{xxx}
	\label{fig:XXXX}
\end{figure}

\todo[author=Alexander]{Maybe an example would be nice.}

One of the largest problems with current \gls{haps} technology (\todo[author=Alexander]{Source?}) is power.
During the day the solar energy is enough to entirely power the HAPS. \todo[author=Alexander]{Source?} However at night solar power is not available forcing the HAPS to get power elsewhere. Current \gls{haps} projects use batteries for overnight power because the distance between the \gls{haps} and the ground is too large for any wired powering.
This presents a problem. The batteries for powering a \gls{haps} for a full night are large and heavy. Using data from \todo[author=Alexander]{Source} a quick estimate of the battery size can be made.

\gls{haps} as network stations have three main sources of power consumption; engines, payload and maintenance. An estimated total of around \SI{115}{kW} is required [\todo[author=Alexander,inline]{Same source as above}]. If we consider an average night to be 12 hours then a \gls{haps} has to be able to store at least \SI{1380}{kWh} in the batteries. A typical lithium ion battery has an energy density of around \SI{5,8}{\kilo\watt\hour\per\kilogram}. \todo[author=Alexander]{Source?}

\begin{equation} \label{eq:HAPS_battery_weight}
	\frac{\SI{1380}{\kilo\watt\hour}}{\SI{350}{\watt\hour\per\kilogram}} = \SI{3942}{\kilogram}
\end{equation}
By removing the batteries alone, without considering housing or power conversion circuitry, \autoref{eq:HAPS_battery_weight} show a \gls{haps} could save almost \SI{4000}{\kilogram} of mass.

This saving makes \gls{haps} an obvious target for wireless power transfer.