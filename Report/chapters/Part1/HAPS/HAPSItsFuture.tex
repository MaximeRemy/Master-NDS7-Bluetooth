\section{HAPS its future}
The first HAPS developed were for military use. It is for two specific tasks, the first one being field surveillance and the second one acting as a relay for any wireless communication. USA pioneer in this domain has already two model of HAPS, Lockheed Martin High Altitude Airship and the project HiSentinel. The ambition of military is to get services akin to a satellite for a lower operational cost.\todo[author=Maxime]{Does a cost comparison would be interesting?)}

However following the same path as the nuclear and Internet, HAPS will be soon recycled for civil use. The current ambition is to use them for communication service delivery. Indeed thanks to its privileged position (HAPS for civil use should be at an altitude of 17 to \SI{22}{km}) it would be able to, ignore any restriction imposed by the region, centralize the network and allow better repartition of the resources. Furthermore its cost can  be potentially less than a ground station, as a suitable site has to be acquired first and a need of relays may arise to spread the network everywhere in the region.
An example of civil HAPS is project LOON financed by Google and has the ambition to let everyone get an access to Internet by launching numerous balloon relay connected to each other and network stations.
As seen on the figure \ref{fig:LoonProject}.

\begin{figure} [h!]
	\centering
	\includegraphics[scale=0.5]{figures/Part1/HAPS/google-project-loon-design}
	\caption{Loon Project design}
	\label{fig:LoonProject}
\end{figure}
