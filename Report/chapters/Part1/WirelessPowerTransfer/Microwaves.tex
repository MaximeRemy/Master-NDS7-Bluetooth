\subsection{Radio waves} \label{sec:radioRadiowaves} 
Radio waves can be used to transfer power wirelessly between antennas.

A frequency converter is used to convert an \gls{ac} signal to the transmitting frequency.
The receiving antenna uses an other frequency converter to convert the high frequency signal to the desired frequency. The two antennas have to point at each other and requires a line of sight in order to achieve maximum transferring efficiency. 
 
The setup for this method is shown in \autoref{fig:RadioAntenna}. 
\begin{figure}[h]
	\centering
	\includegraphics[width=1\linewidth]{figures/Part1/RadioAntenna}
	\caption{Setup with transmitting and receiving antenna of \gls{wpt} using radiowaves.}
	\label{fig:RadioAntenna}
\end{figure}

Radio wave transmission has been found to be very efficient at long distance \citep{TechReport:WPTSheik}.% It is particularly true at \SI{2.45}{\si{\giga\hertz}} which has an efficiency of more than \SI{90}{\percent} \citep{TechReport:WPTSheik}. For power transferring a slotted waveguide antenna type is ideal, because of its high power handling capability which has an efficiency of more than \SI{95}{\percent} \citep{TechReport:WPTSheik}.

In 1987 the Canadian Communications Research Centre (CCRC) researched the idea of having a \gls{wpt} unmanned lightweight airplane act as a relay station for broadcasting services called SHARP. The airplane should circle at an altitude of \SI{21}{\kilo\meter} and stay up months at a time without landing. They successfully constructed a prototype model airplane which flew for 20 minutes at an altitude of \SI{150}{\meter} using a \SI{2.45}{\giga\hertz}, \SI{10}{\kilo\watt} microwave signal \citep{book:shinohara,TechReport:SHARP88}.

The advantage of \gls{wpt} with antennas is the distance for power transfer and the efficiency of the method compared to the other mentioned methods. It also presents a disadvantage that a transmitting antenna can be potentially more dangerous for living beings \citep{manual:ICNIRP_GUIDELINES}, and also needs to have the antennas to be aligned.



%\todo[author=Daniel, inline]{Source for "Reference levels for general public exposure to time-varying electric and magnetic fields": http://www.icnirp.org/cms/upload/publications/ICNIRPemfgdl.pdf}


