\subsection{Inductive coupling} \label{sec:InductiveCoupling}
Inductive coupling consists of two electrically conductive coils placed adjacent to each other at a fixed distance. 

Inductive coupling works by creating a fast changing magnetic field around the power transmitting coil, it is achieved by applying an alternating current to the coil. 
If the power receiving coil is placed within the changing magnetic field, an electric current will be inducted in the power receiving coil, thus transferring electrical energy wirelessly from one coil to the other coil. 
%In order to have the highest efficiency the power receiving coil have to be placed where the magnetic field is strongest.  

An illustration of inductive coupling setup is seen on \autoref{fig:InductiveCouplingSetup}. 
\begin{figure} [h]
	\centering
	\includegraphics[width=0.8\linewidth]{figures/Part1/InductiveCoupling}
	\caption{Setup demonstrating \gls{wpt} using inductive coupling \citep{website:inductive_coupling}.}
	\label{fig:InductiveCouplingSetup}
\end{figure}

Inductive coupling is shown to have a very high efficiency at short range, which dramatically falls when the range is increased \citep{website:Analysis_of_Wireless_Power_Transmission}. To expand the range of the inductive coupling, resonant circuits can be introduced to carry the magnetic field further. This is called resonant coupling.
%
%\todo[author=Daniel, inline]{ Source on topic: https://arxiv.org/ftp/arxiv/papers/1311/1311.5382.pdf}
%
%\todo[author=Daniel, inline]{Source on topic: http://www.jpier.org/PIERM/pierm34/20.13121908.pdf}
%
\subsubsection{Resonant coupling}\label{sec:ResonantCoupling}
Resonant coupling is an expansion on the inductive coupling. It consists of an inductive coupling with one or more resonating circuits introduced into the changing magnetic field.

An illustration of resonant coupling setup with two supplementing coils in between is seen on \autoref{fig:ResonantCouplingSetup}. 
\begin{figure} [h]
	\centering
	\includegraphics[width=0.8\linewidth]{figures/Part1/ResonantInductiveCoupling}
	\caption{Resonant coupling setup with two supplementing coils between \citep{website:resonant_inductive_coupling}.}
	\label{fig:ResonantCouplingSetup}
\end{figure}

Resonant coupling can have very high efficiency at medium range, but like with inductive coupling it exponentially decreases when the range is increased, the addition of more coils will however improve efficiency and allow for a longer transferring range.

A team of MIT students showed that it is possible to transfer energy over a distance of 2 meters, with an efficiency of 40\%, by using this method \citep{website:WiTricity_Highly_Resonant_Wireless_Power}. Subsequently Intel reproduced the setup and shown an efficiency of 75\% at a shorter distance \citep{TechReport:WPTSheik}.
