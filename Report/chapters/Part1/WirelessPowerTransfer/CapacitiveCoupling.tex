\subsection{Capacitive coupling}\label{sec:capacitiveCoupling}
The power is transferred by having four conducting plates, two at the power transmitting part and two at the power receiving part, placed in pairs of two adjacent to one and other.

The capacitive coupling makes use of the E-field in order to transfer power wirelessly. The method utilizes the principle of a capacitor using the E-field to cause a current to run between the receivers to plates, and thereby transfer power wirelessly.

On \autoref{fig:capacitiveCoupling} a illustration of a capacitive coupling setup is seen \citep{sci_article:capacitively_coupled_power_transfer}.
\begin{figure}[h]
	\centering
	\includegraphics[width=0.6\linewidth]{figures/Part1/CapacitiveCouplingPDF}
	\caption{Capacitive Coupling setup \citep{website:capacitive_bipolar}.}
	\label{fig:capacitiveCoupling}
\end{figure}

This method relies on plates being close enough to each other to behave as capacitors. Therefore this method is not suitable for energy transfer over longer distances.
%\todo[author=Daniel, inline]{Source on topic: http://ieeexplore.ieee.org/stamp/stamp.jsp?arnumber=6566413 }

%The Capacitive coupling setup can be modelled as a circuit, where the pairs of power transmuting plates can be considered as capacitors. 
%
%circuit diagram here. 
%
%writing up the voltage division for the circuit yields: 
%\begin{equation}
%V_{out} = \frac{R}{R + \frac{1}{s \cdot C_1 } + \frac{1}{s \cdot C_2 }}  \cdot V_{in} 
%\end{equation}
%\begin{equation}
%V_{out} = \left| \frac{R}{R + \frac{1}{j \cdot \omega \cdot C_1 } + \frac{1}{j \cdot \omega \cdot C_2 }}  \cdot V_{in} \right|
%\end{equation}
%In order to have the highest possible output voltage, the capacitance of $C_1$ and $C_2$ should have as high as possible. The frequency should be as high as possible. Furthermore the load, R should be very big compared to $\frac{1}{s \cdot C_1,2}$. 
%
%The size of C_1 and C_2 is determined by the physical dimensions of the  two pairs of plates. 
%\begin{equation}
%C = \frac{\varepsilon \cdot A}{d}
%\end{equation}
%In order to have a high capacity A should be larger then d.  
%
%Assuming the four plates is going to be equal in size, the equations combined yields: 
%\begin{equation}
%V_{out} = \left| \frac{R}{R + \frac{1}{j \cdot \omega \cdot \frac{\varepsilon \cdot A}{d} } + \frac{1}{j \cdot \omega \cdot \frac{\varepsilon \cdot A}{d} }}  \cdot V_{in} \right|
%\end{equation}
%
%isolation for A 
%\begin{equation}
%A = \left| \frac{2 \cdot d \cdot V_{out}}{R \cdot \varepsilon \cdot j \cdot \omega \cdot (V_{in} - V_{out}) } \right|
%\end{equation}
%\begin{equation}
%A = \left| \frac{2 \cdot 20 \cdot 50}{R \cdot \varepsilon \cdot j \cdot \omega \cdot (V_{in} - V_{out}) } \right|
%\end{equation}






