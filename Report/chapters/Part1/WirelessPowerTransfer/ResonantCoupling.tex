\subsubsection{Resonant coupling}\label{sec:ResonantCoupling}
Resonant coupling is an expansion on the inductive coupling. It consists of an inductive coupling with one or more resonating circuits introduced into the changing magnetic field.

An illustration of resonant coupling setup with two supplementing coils in between is seen on \autoref{fig:ResonantCouplingSetup}. 
\begin{figure} [h]
	\centering
	\includegraphics[width=0.8\linewidth]{figures/Part1/ResonantInductiveCoupling}
	\caption{Resonant coupling setup with two supplementing coils between \citep{website:resonant_inductive_coupling}.}
	\label{fig:ResonantCouplingSetup}
\end{figure}

Resonant coupling can have very high efficiency at medium range, but like with inductive coupling it exponentially decreases when the range is increased, the addition of more coils will however improve efficiency and allow for a longer transferring range.

A team of MIT students showed that it is possible to transfer energy over a distance of 2 meters, with an efficiency of 40\%, by using this method \citep{website:WiTricity_Highly_Resonant_Wireless_Power}. Subsequently Intel reproduced the setup and shown an efficiency of 75\% at a shorter distance \citep{TechReport:WPTSheik}.
