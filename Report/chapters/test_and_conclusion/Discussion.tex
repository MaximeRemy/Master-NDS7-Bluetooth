\chapter{Discussion} \label{ch:Discussion}
In \autoref{sec:designOscillator} it is chosen to use a temperature sensitive \gls{vco}, without compensating circuitry such as a \gls{pll}, as the local oscillator in the mixer circuit and the transmitter. When the potentiometer is used to set the right tune frequency, the system is stable for about \SI{60}{\second} before the frequency has drifted outside of the acceptable area. The tests in \autoref{ch:testing} was therefore restricted to the ground with signal generators replacing the oscillators. The tests can however still be used to measure the performance of the rest of the system.

Tests showed that the rest of the system was functional, however with less precision and speed than the stated requirements. The system was not tested for its planned use case, due to being limited to the ground. The choices made during the project will now be discussed, and suggestions for improvements or alternatives will be mentioned.

It was described in \autoref{sec:TrackingTheDrone} (\autoref{sec:pre:trackingConclusion}) that the signal distance difference which utilises the phase difference of the received signal to determinate the \gls{aoa} to track the drone. This turned out to work with a precision of \SI{1.8}{\milli\radian} which better than the requirement stated in \autoref{sec:systemRequiremetnsTracking} \autoref{req:track_precision}. Offsets and limited field of view in the estimated \gls{aoa} were observed, these could be due to incorrect phase-shifting of the \SI{90}{\degree} shifted antenna or noise due the directionality of the receiving antennas, limiting the received signal. The receiver antennas also had lower gain than expected for such a directional antenna, thus further lowering the received signal strength.

Phase calculation is done by multiplying two signals of the same frequency. Followed by a low-pass filtering to single out the phase difference. This choice imposed a requirement for one of the signals to be \SI{90}{\degree} phase shifted. The method was shown working, but its current implementation is not very noise resistant, since the result of the calculation is sensitive to all frequency components of the signal. The cycle time of the controller was determined to be \SI{161}{\milli\second} as opposed to the assumed \SI{5}{\milli\second}, due to the phase detection taking a long time to compute. This results in a system response that deviates from expectations leading to \autoref{fig:implementation:stepResponse}.

Assuming that a stable beacon signal frequency can be achieved. Further development could include a narrow digital bandpass filter, to single out the beacon signal and optimizations to reduce the computation time. It might be worthwhile to consider other processing boards or other methods of phase detection to improve calculation time and robustness in respect to noise. If a stable beacon signal frequency can be achieved, a single point of a \gls{dft} analysis might be sufficient, and the number of samples could be adjusted to reduce the spectral leakage happening due to the signal not being harmonic for the sampled period. Further investigation is however necessary to determine whether this approach will offer an improvement over the current one. A hardware approach e.g. an \texttt{XOR}-gate and filtering could also be investigated as an alternative solution, thus lowering the cycle time of the controller significantly.

Another important subject for discussion is the choice of frequency. The chosen frequency band has a maximum duty cycle of 0.1\% \citep{web:RegulationFreq869}, meaning that transmission of a continuous signal is not allowed. For a end product, the choice of frequency or method will have to be altered, to comply with applicable laws and restrictions.

The integration of the sub-modules of the tracking system also suffers in some parts. The maximum expected signal level has not been evaluated in respect to the chosen reference voltage of the \gls{adc}. This can be the source of a lower than desired \gls{snr} after sampling. Ideally the reference voltage would be adjusted depending on the current signal level, to ensure maximum effective resolution. It was also observed that the signal from the \gls{lo} propagates into the receiving antenna, and thereby adds more frequency components to the received signal. Shielding of the \gls{lo} might be a solution for this problem.

After adjusting for the offset in the tracking module, the full feedback control system works. The azimuth and elevation angles are adjusted to point towards the transmitting antenna. The movement is however not smooth, has steady-state errors and unexpected overshoots. This might be caused by problems with the controller. 

The controller has two problems. The first problem is that it is designed with the assumption of a cycle time of \SI{5}{\milli\second} that turned out to be \SI{160}{\milli\second}, causing unexpected behaviour. The second problem is its compensation not being properly adjusted, causing the DC motor to stop rotation before the right angle is achieved, showing one of the downsides of this approach. Furthermore the controller design did not take the expected acceleration and maximum velocity of the drone into consideration. To ensure that the system stays within the required minimum error at all times, the movement of the drone must be further analysed and evaluated during controller design. 

To assist the design, the antenna stand model should be tested towards the physical system and be adjusted for the best possible fit, to make simulations more precise, thus ensuring a better correlation between the simulations and the physical system. This would ease the design of a matched controller.

To improve the system, the cycle time can be lowered leading to a more stable and faster reacting system, a switch from a \gls{pd} to a \gls{pid} controller can result in smaller errors during movement, and expanding the system to a cascade control system might offer some additional improvements. These could therefore be topics for further development of the system.

The system can be expanded to include a communications network between the drone and ground station, to communicate energy levels and monitor the transmitted power. This would also open up for the possibility of using differential GPS as supplement or replacement for the current tracking solution, where the drones position could be communicated over the communications network.

The system with improvement discussed previously could then track a drone in the air efficiently. The next step would then be to find an efficient method to wirelessly transmit power to the drone. As said in \autoref{sec:ConclusionWPT} the most efficient method of \gls{wpt} is to use energy carrying microwaves.

Microwaves can potentially harm living beings. If microwaves are to be used to transmit power, it is necessary to set up counter measures to avoid accidents. The charging station should ensure that no living beings are is risk of harm before starting its transmission of power. 
%
%The microwaves method of \gls{wpt} requires knowledge of the distance to from the station to the drone, in order of achieve maximum efficiency. If the drone is at the wrong distance, the beam can either miss its maximal efficiency or go past the drone, meaning it can affect anything behind it. To ensure maximum efficiency the system should be expanded to detect the distance between the drone and the station. 




