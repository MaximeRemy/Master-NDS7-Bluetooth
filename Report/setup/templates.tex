%Equations
%One equation with description
\begin{equation}
F = m \cdot a \addunit{\newton}
\end{equation}
\startexplain
	\explain{F is the net force applied}{\si{\newton}}
	\explain{m is the mass of the body}{\si{\kilogram}}
	\explain{a is the body's acceleration}{\si{\meter\per\second\squared}}
\stopexplain

%Two equations without description
\begin{subequations} \label{eq:tech_ToA}
	\begin{flalign}
		&\varphi = \cos^{-1} \left( \frac{y}{x} \right) \\
		&y = \Delta \tau \cdot c \\
		&\theta \approxeq \SI{90}{\degree} - \varphi \\
		&\theta \approxeq \SI{90}{\degree} - \cos^{-1} \left( \frac{ \Delta\tau \cdot c}{x} \right)
	\end{flalign}
\end{subequations}


%%%%%%% Variable naming rules %%%%%%%
% DC = 		$v_A$
% AC = 		$v_a(t)$
% Both = 	$v_a(t)$
% $L{v_a(t)} = V_a(s)$
% j = complex operator
% i = current
% v = voltage
% sin = $\sin$
% cos = $\cos$
% log = $\log$
% arcsin = \arcsin(v)
% arcsin^2 = \arcsin^2(v)
% arcsin(v^2+u^2) = \arcsin((v+u)^2)

%%%%%% Adding a glossary entry %%%%%%%
%%In glossary. tex file
\newacronym{e}{E}{Example} %{label}{initials}{Full words with initial in capitals}
%%In the report whenever you want to use it
\gls{e}



%Tables
%Big table
\begin{table}[h]
	\centering
	\caption{controller commands and explanation of these.}
	
	\begin{tabularx}{\textwidth}{lX}
		\textit{Command}		&	\textit{Explanation} \\ \rowcolor{lightGrey} \toprule
		Reset					&	Is used to re-initialize the system and the internal registers.\\
		Write					&	Is used to write to a register in a specific address. Can continue to be sent to sequential registers.\\ \rowcolor{lightGrey}
		Read					&	Reads data from a specified register. Shifts out data of the register and increments to next register.\\
		Bit Modify				&	Sets, modifies or clears a specified bit in status and control registers.\\ \rowcolor{lightGrey}
		Read Status				&	Allows instruction access to status bits.\\
	\end{tabularx}
	\caption*{\citep{MCP2510}}
\end{table}\label{tab:mcp_cmd_expl}





%Small table

\begin{table}[h]
	\centering
	\caption{Summary of the different methods of tracking the drones direction.}\label{tab:trackdirectconcl}
	\begin{tabular}{ll}
		\textit{Method} 				& \textit{Summery}  	\\ \toprule \rowcolor{lightGrey}
		Optical tracking				& Impractical \\
		Phase difference 				& Good solution \\ \rowcolor{lightGrey}
		Signal strength 				& Very noise prone and not precise \\
		Time of arrival					& Requires a high sampling frequency and sensitive electronics \\
	\end{tabular}
\end{table}



%Figure 
\begin{figure} [htbp]
	\centering
	\includegraphics[width=1\linewidth]{CAN_frame_without_bitstuff}
	\caption{Example of complete \gls{can} frame without bit stuffing \citep{CAN-Bus-frame}.}
	\label{fig:frame_wo_bitstuff}
\end{figure}